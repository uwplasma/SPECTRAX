% Companion documentation for landau_hermite_jax.py
% This file is intended to be self-contained and pedagogic. It is not a journal manuscript.
\documentclass[11pt]{article}

\usepackage[margin=1in]{geometry}
\usepackage{microtype}
\usepackage{amsmath,amssymb,mathtools}
\usepackage{bm}
\usepackage{siunitx}
\usepackage{booktabs}
\usepackage{graphicx}
\usepackage{subcaption}
\usepackage{xcolor}
\usepackage{hyperref}
\usepackage{enumitem}
\usepackage{fancyvrb}

\hypersetup{
  colorlinks=true,
  linkcolor=blue!55!black,
  urlcolor=blue!55!black,
  citecolor=blue!55!black
}

\graphicspath{{./}{tests_landau_hermite/}{tests_landau_hermite/latest/}}

\title{Companion Notes for \texttt{landau\_hermite\_jax.py}:\\
Fast Landau--Hermite Collisions via SOE\texorpdfstring{$\to$}{->}MPO/TT (JAX-first)}
\author{(auto-generated companion for the standalone script)}
\date{\today}

\newcommand{\code}[1]{\texttt{#1}}
\newcommand{\dd}{\mathrm{d}}
\newcommand{\RR}{\mathbb{R}}

\begin{document}
\maketitle

\begin{abstract}
This document explains the numerics, implementation, and test suite used by the standalone script
\code{landau\_hermite\_jax.py}. The script implements a fast evaluation of the spatially homogeneous
Landau (Fokker--Planck) collision operator in a 3D tensor-product Hermite basis, using a separable
quadrature (``SOE'' style separation), tensor-product operator factorizations (``MPO'' viewpoint),
and optional tensor-train (TT) compression (currently used in the NumPy debug path; the JAX path
relies on JIT+vectorized contractions).

The goal is pedagogic completeness: what the code computes, how it is normalized, how it is
time-stepped, what ``linearized'' means in this context, why the tests are designed the way they are,
and how to interpret the produced figures (Fig.~1/Fig.~2 panels and the test plots).
\end{abstract}

\tableofcontents
\newpage

\section{Overview: equations, goals, and where the cost lives}

This companion is meant to read like a guided tour from first principles to the specific fast
implementation in \code{landau\_hermite\_jax.py}. The structure is:
\begin{enumerate}[nosep]
  \item What equation we solve and what is measured (physics).
  \item How we discretize velocity with a Hermite basis (numerics).
  \item Why the nonlinear Landau operator is expensive (bottlenecks).
  \item How separability (SOE), tensor-product contractions (MPO), and optional compression (TT) remove
  the bottlenecks.
  \item How the script time-steps, tests itself, and produces Fig.~1/Fig.~2 panels.
\end{enumerate}

\subsection{The equations solved by the script}
The Landau collision operator describes collisional relaxation in velocity space. In this project we
consider the \emph{spatially homogeneous} setting, where the distribution functions depend only on
velocity $\bm{v}\in\RR^3$ and time $t$.

\paragraph{One species.}
For a single species with distribution $f(\bm{v},t)$ we solve the ODE/PDE
\begin{align}
  \partial_t f = \nu\,Q(f,f),
\end{align}
where $\nu$ is a collision frequency (a scalar in this script), and $Q$ is the nonlinear Landau
collision operator.

\paragraph{Two species.}
For two species $a$ and $b$ we solve
\begin{align}
  \partial_t f_a &= \nu_{ab}\,Q_{ab}(f_a,f_b),\\
  \partial_t f_b &= \nu_{ba}\,Q_{ba}(f_b,f_a),
\end{align}
with parameters chosen so that the \emph{total} invariants (summed across species) are conserved.

\subsection{What we are trying to compute}
The script does \emph{not} discretize physical space; it only evolves velocity-space dynamics.
Concretely, it:
\begin{itemize}[nosep]
  \item represents each $f(\bm{v},t)$ in a tensor-product Hermite basis in the normalized velocity
  variables $(x,y,z)=(v_x/v_{\mathrm{th}},v_y/v_{\mathrm{th}},v_z/v_{\mathrm{th}})$;
  \item evolves the corresponding coefficient tensor in time using an explicit ODE integrator;
  \item computes diagnostics (density, momentum, energy, temperatures, anisotropy) and produces the
  Fig.~1/Fig.~2 panels;
  \item optionally (\code{--run\_tests}) runs a correctness/performance suite that produces additional
  plots and timing data.
\end{itemize}

\subsection{Why this is hard: the nonlinear Landau bottleneck}
At a high level, the nonlinear Landau operator is expensive for two reasons:
\begin{enumerate}[nosep]
  \item \textbf{Nonlocality in velocity:} $Q(f,f)$ involves an integral over $\bm{v}'$ with a
  long-range Coulomb kernel. Naively, evaluating this for many velocity degrees of freedom is
  comparable to a dense integral operator.
  \item \textbf{Mode coupling:} in a spectral representation, each output coefficient couples to many
  input coefficients. A naive coefficient-space implementation resembles a high-rank tensor
  contraction with $\mathcal{O}(p^6)$--$\mathcal{O}(p^7)$ work (depending on formulation), where
  $p=n_{\max}+1$ per velocity dimension.
\end{enumerate}
The core purpose of the ``SOE$\to$MPO/TT'' machinery is to replace these naive dense couplings by a sum
of \emph{separable} tensor-product contractions, so the cost is dominated by repeated 1D contractions
that are friendly to JIT compilation and cache/accelerator hardware.

\subsection{Roadmap of the fast algorithm (one paragraph)}
In the Hermite formulation used here, the coefficient formulas can be reorganized so that many factors
depend only on \emph{one-dimensional} indices in $x$, $y$, and $z$, while the Coulomb kernel moments
enter through a scalar factor that can be written as an integral over $s\in[0,1]$. Approximating that
scalar factor by Gauss--Legendre quadrature yields a \emph{finite sum of separable terms} (called
``SOE'' in this code base). Each term becomes a tensor-product operator acting along $x$, $y$, and $z$,
which can be applied by three successive 1D contractions (an ``MPO'' viewpoint). For larger
$n_{\max}$, some intermediate tensors can optionally be compressed using a tensor-train (TT) format
(currently in the NumPy debug path).

\section{How to run the code (quickstart and reproducibility)}

\subsection{What the script produces}
Running \code{python landau\_hermite\_jax.py} (defaults) writes:
\begin{itemize}[nosep]
  \item \code{Fig1\_panel.pdf} and \code{Fig1\_panel.png}: a 2$\times$2 diagnostic panel for a
  one-species relaxation problem and a two-species equilibration problem.
  \item \code{Fig2\_panel.pdf} and \code{Fig2\_panel.png}: same dynamics/parameters as Fig.~1, but
  with a different, strongly non-Maxwellian initial condition (IC) designed to remain nonnegative on
  diagnostic grids.
  \item Optionally (if \code{--run\_tests}), a timestamped directory under
  \code{tests\_landau\_hermite/run\_YYYYmmdd\_HHMMSS/} containing plots, CSV, and JSON summaries. The
  latest run is also linked as \code{tests\_landau\_hermite/latest}.
\end{itemize}

\subsection{Backend}
The script has two backends:
\begin{itemize}[nosep]
  \item \textbf{JAX backend} (\code{--backend jax}, default): uses \code{jax.jit}, \code{vmap}/\code{einsum},
  and \code{lax.scan} time stepping for performance and scalability.
  \item \textbf{NumPy backend} (\code{--backend numpy}): for debugging and for certain explicit checks.
\end{itemize}
Both backends use the \emph{same} precomputed coefficient tables and the same fast SOE$\to$MPO contraction
structure; the difference is whether contractions are executed with NumPy or JAX/XLA.

\subsection{CLI options}
The script self-documents its CLI via \code{-h}. For day-to-day use, the most important options are:
\begin{itemize}[nosep]
  \item \code{--nmax}: Hermite truncation ($p=n_{\max}+1$ modes per dimension).
  \item \code{--backend \{jax,numpy\}}: performance vs debugging.
  \item \code{--Q}, \code{--maxK}: accuracy/cost knobs for the separated kernel evaluation.
  \item \code{--linearized on/off}: include linearized overlays (computed within the same script).
  \item \code{--run\_tests}: generate the test suite plots and timing summaries in \code{tests\_landau\_hermite/}.
\end{itemize}
The full help text is included in Appendix~\ref{app:cli}.

\section{Physical model: Landau collision operator (first principles)}

\subsection{Continuous operator (context)}
For species $a$ colliding with species $b$, the Landau collision operator can be written
(in one common form) as
\begin{align}
  Q_{ab}(f_a,f_b)(\bm{v})
  = \frac{\partial}{\partial v_i}\int_{\RR^3} U_{ij}(\bm{u})
  \left[
    f_b(\bm{v}')\,\frac{\partial f_a(\bm{v})}{\partial v_j}
    - \frac{m_a}{m_b}\,f_a(\bm{v})\,\frac{\partial f_b(\bm{v}')}{\partial v'_j}
  \right]\dd \bm{v}',\qquad \bm{u}=\bm{v}-\bm{v}',
\end{align}
where repeated indices are summed over $i,j\in\{1,2,3\}$ and, for Coulomb interactions,
\begin{align}
  U_{ij}(\bm{u}) = \frac{\delta_{ij}|\bm{u}|^2-u_i u_j}{|\bm{u}|^3}
  \;=\;\frac{1}{|\bm{u}|}\left(\delta_{ij}-\frac{u_i u_j}{|\bm{u}|^2}\right).
\end{align}
In this standalone script we absorb all dimensional prefactors (charges, Coulomb logarithm, etc.)
into the scalar collision frequencies $\nu_{ab}$ and $\nu_{ba}$ that multiply $Q_{ab}$ in time.
This operator is:
\begin{itemize}[nosep]
  \item \textbf{bilinear} in $(f_a,f_b)$,
  \item \textbf{conservative}: it preserves total number, total momentum, and total energy
  (appropriate combinations across species),
  \item \textbf{entropy producing}: for the \emph{nonlinear} operator, relative entropy to a local Maxwellian
  is non-increasing in time (H-theorem).
\end{itemize}
Standard references include Landau (1936), Rosenbluth--MacDonald--Judd (1957), and modern expositions such
as Helander \& Sigmar (2002) and Villani (2002).

\subsection{Rosenbluth potentials viewpoint (intuition)}
One way to interpret the Landau operator is via \emph{Rosenbluth potentials}, which rewrite the
nonlocal integral in terms of two scalar convolution-type potentials $G$ and $H$ of $f_b$.
This viewpoint makes clear why the operator is simultaneously:
\begin{itemize}[nosep]
  \item \textbf{nonlocal} (it depends on velocity-space potentials), and
  \item \textbf{differential} in $\bm{v}$ (it is a divergence of drift+diffusion terms).
\end{itemize}
The present code does not explicitly solve Poisson equations for potentials; instead it evaluates the
same mathematical content directly in Hermite space via structured coefficient formulas. Still, it is
helpful to keep the Rosenbluth picture in mind: ``build potentials of $f_b$'' then ``apply them to
$f_a$''.

\subsection{What the script discretizes}
The code discretizes \emph{velocity} in a tensor-product Hermite basis and evolves the corresponding
expansion coefficients in time for spatially homogeneous relaxation. All spatial dependence is absent:
the ODE is in coefficient space.

\section{From the continuous equation to a coefficient-space ODE}
\subsection{Step 1: nondimensionalize velocity}
For each species we pick a thermal speed $v_{\mathrm{th}}$ and define dimensionless velocity
coordinates $(x,y,z)=\bm{v}/v_{\mathrm{th}}$. This isolates temperature/mass dependence into a few
scalars and makes the basis and diagnostics simple.

\subsection{Step 2: expand $f$ in a tensor-product Hermite basis}
We represent the distribution in a truncated basis:
\begin{align}
  f(x,y,z,t) \approx \sum_{\alpha,\beta,\gamma=0}^{n_{\max}}
  f[\alpha,\beta,\gamma](t)\,\psi_{\alpha}(x)\psi_{\beta}(y)\psi_{\gamma}(z),
\end{align}
and evolve the coefficient tensor $f[\alpha,\beta,\gamma](t)$.

\subsection{Step 3: project $Q(f_a,f_b)$ onto the basis}
Applying Galerkin projection (take inner products with test functions) yields a closed ODE system:
\begin{align}
  \frac{\dd}{\dd t} f_k(t) = \sum_{k_a,k_b} C^{ab}_{k,k_a,k_b}\; f_{a,k_a}(t)\;f_{b,k_b}(t),
\end{align}
where $k$ is a multi-index for $(\alpha,\beta,\gamma)$ and $C^{ab}$ is a (very large) dense bilinear
coefficient tensor encoding the collision physics.

\paragraph{What is $C^{ab}_{k,k_a,k_b}$ \emph{exactly}?}
The script ultimately implements the coefficient-level expression obtained by projecting the Landau form
onto the (species-dependent) Hermite basis. Using normalized velocities
$\overline{\bm v}_a=\bm v/v_{\mathrm{th},a}$ and $\overline{\bm v}'_b=\bm v'/v_{\mathrm{th},b}$, and letting
$\psi^a_k(\overline{\bm v}_a)$ denote the 3D tensor-product basis functions for species $a$ (with the 1D
definition in Sec.~5), one convenient coefficient expression is:
\begin{align}
  C^{ab}_{k,k_a,k_b}
  =-\frac{\nu_{ab}}{2}\int_{\RR^3}\!\!\int_{\RR^3}
  \frac{\partial \psi^{a,k}}{\partial \overline v_{i,a}}(\overline{\bm v}_a)\;
  \overline U_{ij}(\overline{\bm u}_a)\;
  \left[
    \psi^b_{k_b}(\overline{\bm v}'_b)\,\frac{\partial \psi^a_{k_a}}{\partial \overline v_{j,a}}(\overline{\bm v}_a)
    - \frac{m_a}{m_b}\frac{v_{\mathrm{th},a}}{v_{\mathrm{th},b}}\,
    \psi^a_{k_a}(\overline{\bm v}_a)\,\frac{\partial \psi^b_{k_b}}{\partial \overline v'_{j,b}}(\overline{\bm v}'_b)
  \right]\dd \overline{\bm v}_a\,\dd \overline{\bm v}'_b,
  \label{eq:Cabkkakb_companion}
\end{align}
where $\overline{\bm u}_a=\overline{\bm v}_a-(v_{\mathrm{th},b}/v_{\mathrm{th},a})\,\overline{\bm v}'_b$ and
$\overline U_{ij}$ is the same Landau tensor kernel written in the normalized variables (so it depends only
on the relative velocity argument). Equation \eqref{eq:Cabkkakb_companion} is the ``source of truth''
equation we refer back to when explaining every fast step (SOE, MPO contractions, TT compression) below:
all of the algorithmic machinery is about evaluating \eqref{eq:Cabkkakb_companion} \emph{without} forming a
dense 6D coupling tensor.

\subsection{Where the cost comes from (the bottleneck, explicitly)}
If we denote $p=n_{\max}+1$ and $N=p^3$ coefficients per species, then a direct evaluation of the
above dense bilinear form would be:
\begin{itemize}[nosep]
  \item \textbf{memory-prohibitive}: storing $C^{ab}$ is $\mathcal{O}(N^3)$;
  \item \textbf{compute-prohibitive}: contracting it is also $\mathcal{O}(N^3)$ operations.
\end{itemize}
Even if one never explicitly forms $C^{ab}$, a naive implementation of the nested sums typically
scales worse than $\mathcal{O}(p^6)$ in practice. The entire purpose of the fast machinery in this
project is to compute the same RHS using only:
\begin{itemize}[nosep]
  \item precomputed \emph{1D tables} (cheap to build and reuse), and
  \item a small number of repeated \emph{tensor-product contractions} (cheap to evaluate when fused).
\end{itemize}

\section{Normalization, basis, and coefficient tensor representation}

\subsection{Thermal normalization}
Each species has a reference thermal speed $v_{\mathrm{th}}$ and mass $m$. The code uses normalized
velocity coordinates
\begin{align}
  x = v_x / v_{\mathrm{th}},\qquad y = v_y / v_{\mathrm{th}},\qquad z = v_z / v_{\mathrm{th}}.
\end{align}
The convention used throughout is
\begin{align}
  v_{\mathrm{th}} = \sqrt{\frac{2T_{\mathrm{eq}}}{m}}
  \quad \Longleftrightarrow \quad
  T_{\mathrm{eq}} = \frac{m v_{\mathrm{th}}^2}{2},
\end{align}
where $T_{\mathrm{eq}}$ is the equilibrium temperature corresponding to the chosen normalization.

\subsection{Hermite basis used in the code}
The script uses a Hermite--Gaussian basis in each dimension:
\begin{align}
  \psi_n(x) = \frac{1}{\sqrt{\pi}}\frac{H_n(x)}{\sqrt{2^n n!}}\,e^{-x^2},
\end{align}
where $H_n$ is the physicists' Hermite polynomial. The code implements $\psi_n$ in \code{psi\_1d}.

\paragraph{Why Hermite? (intuition)}
Hermite--Gaussian bases are well matched to collisional kinetic problems because:
\begin{itemize}[nosep]
  \item Maxwellians are Gaussians, so the equilibrium state is captured very compactly
  (in this script, the equilibrium Maxwellian corresponds to a single nonzero coefficient
  $f[0,0,0]$ for each species).
  \item Multiplication by $x$ and differentiation in $x$ act by \emph{ladder relations} that shift $n$
  by $\pm 1$, making many velocity-moment and derivative operations sparse in coefficient space.
  \item A tensor-product basis makes 3D operations separable across $(x,y,z)$, which is essential for
  the MPO-style contractions used later.
\end{itemize}

\paragraph{Orthogonality and normalization.}
With the above convention, $\{\psi_n\}_{n\ge 0}$ are orthonormal on $\RR$:
\begin{align}
  \int_{\RR} \psi_n(x)\,\psi_m(x)\,\dd x = \delta_{nm}.
\end{align}
The 3D tensor basis $\psi_{\alpha}(x)\psi_{\beta}(y)\psi_{\gamma}(z)$ is then orthonormal on $\RR^3$,
so inner products and many diagnostics reduce to sums over low-order coefficients.

\paragraph{Ladder identities (used implicitly).}
The Hermite functions satisfy identities of the schematic form
\begin{align}
  x\,\psi_n(x) &\propto \sqrt{n+1}\,\psi_{n+1}(x) + \sqrt{n}\,\psi_{n-1}(x),\\
  \partial_x \psi_n(x) &\propto \sqrt{n+1}\,\psi_{n+1}(x) - \sqrt{n}\,\psi_{n-1}(x),
\end{align}
up to fixed constants determined by the chosen normalization. In coefficient space, these become sparse
``shift'' operators. The fast Landau operator construction uses these structured shifts when building
the 1D tables.

\subsection{Tensor coefficient representation}
For a truncation parameter $n_{\max}$, define $p=n_{\max}+1$. The coefficient tensor is
\begin{align}
  f[\alpha,\beta,\gamma], \quad \alpha,\beta,\gamma\in\{0,\dots,n_{\max}\},
\end{align}
so that
\begin{align}
  f(x,y,z) \approx \sum_{\alpha,\beta,\gamma=0}^{n_{\max}} f[\alpha,\beta,\gamma]\,
  \psi_{\alpha}(x)\psi_{\beta}(y)\psi_{\gamma}(z).
\end{align}
The code keeps this 3D ``cube'' representation (shape $(p,p,p)$) throughout to avoid flattening overhead
and to make tensor contractions explicit and JIT-friendly.

\subsection{Moment diagnostics and invariants}
From the Hermite coefficients, the script extracts:
\begin{itemize}[nosep]
  \item density $n$,
  \item momentum $\bm{P}$,
  \item total kinetic energy $W$,
  \item temperature $T = \frac{2}{3}\frac{W}{n}$,
  \item anisotropy measure $A = (T_z-T_x)/T_{\mathrm{avg}}$.
\end{itemize}
These are computed from low-order coefficients (see \code{invariants\_from\_tensor} and
\code{temperature\_components\_hat\_from\_tensor}). Conservation of $(n,\bm{P},W)$ (or total invariants
for 2 species) is one of the most stringent correctness checks for collision operators.

\section{Fast collision evaluation: SOE\texorpdfstring{$\to$}{->}MPO/TT}

\subsection{What we are accelerating: the coefficient-space collision RHS}
Equation \eqref{eq:Cabkkakb_companion} is a double velocity integral with products and derivatives of
basis functions and a nonlocal kernel. After Galerkin projection, this becomes a dense bilinear form
in coefficient space (Sec.~4), but the \emph{physics} structure has not disappeared; it is encoded in
very specific combinatorial factors coming from Hermite identities and a small family of kernel
moments.

\subsection{Acronyms and the high-level idea}
We will use three acronyms throughout:
\begin{itemize}[nosep]
  \item \textbf{SOE} (``separated sum'' / ``separable quadrature''): replace one blocking scalar factor by a finite sum of separable terms using quadrature; see \cite{BeylkinMonzon,GolubWelsch}.
  \item \textbf{MPO} (matrix product operator): apply a tensor-product (Kronecker) operator to a tensor by successive 1D contractions; see \cite{Schollwoeck2011,Orus2014}.
  \item \textbf{TT} (tensor train): compress 3D tensors via low-rank cores (same format as MPS); see \cite{Oseledets2011,Hackbusch2012}.
\end{itemize}
The overall goal is: turn the dense multi-index couplings into \emph{(i)} precomputed 1D tables and
\emph{(ii)} a small number of repeated tensor-product contractions.

\subsection{Where the main bottleneck lives: one scalar factor that couples $x,y,z$}
Using ladder identities (derivatives) and Hermite product expansions (products), most of the basis
algebra in Eq.~\eqref{eq:Cabkkakb_companion} reduces to 1D tables such as $P^{k'}_{n,m}$ (stored as
\code{P1D} in the script). What remains is the kernel moment structure. In the derivation used by the
original project, the Coulomb kernel produces scalar moments
\begin{align}
  I_{K}=\int_{\RR^3}\frac{\psi_{K}(\bm u)}{|\bm u|}\,\dd \bm u,\qquad K=(2a,2b,2c),
\end{align}
that can be written in closed form as
\begin{align}
  I_{K}=\frac{(-1)^{A}}{\sqrt{\pi}(2A+1)}\,
  \frac{\sqrt{(2a)!(2b)!(2c)!}}{2^{A-1}\,a!\,b!\,c!},\qquad A=a+b+c.
  \label{eq:IK_companion}
\end{align}
Everything in \eqref{eq:IK_companion} is separable across dimensions \emph{except} the single rational
factor $1/(2A+1)$, because $A=a+b+c$ is an \emph{additive} multi-index. Unlocking separability of this
factor is the SOE step.

\subsection{SOE (separable quadrature): separating the kernel factor}
The key identity is the Beta-function integral
\begin{align}
  \frac{1}{2A+1}=\int_{0}^{1} s^{2A}\,\dd s.
  \label{eq:boys0}
\end{align}
With $A=a+b+c$,
\begin{align}
  s^{2A}=s^{2a}\,s^{2b}\,s^{2c},
\end{align}
so for each fixed $s$ the dependence is separable in $(a,b,c)$. Approximating \eqref{eq:boys0} by
$Q$-point Gauss--Legendre quadrature on $[0,1]$ yields a finite separated sum:
\begin{align}
  \frac{1}{2(a+b+c)+1}\approx \sum_{q=1}^{Q} w_q\;\Big(s_q^{2a}\Big)\Big(s_q^{2b}\Big)\Big(s_q^{2c}\Big).
  \label{eq:soe}
\end{align}
This is the \textbf{SOE} step: one coupled 3D dependence becomes a sum of $Q$ fully separable terms.
In this code base, \code{Q} is small (default 8), so the $q$-sum is efficient and can be vectorized
(\code{vmap}) or looped with compiled control flow (\code{lax.fori\_loop}) in JAX.

\paragraph{Accuracy knob.}
At fixed $n_{\max}$, increasing $Q$ improves the kernel-moment accuracy (hence Maxwellian residuals).
The script includes a small ``auto-$Q$'' heuristic (enabled by default) and the test suite includes
$(Q,\texttt{maxK})$ convergence plots for reviewer-facing confidence.

\subsection{Hankel-like tables: precomputing the remaining 1D dependence}
After SOE separation, what remains are purely 1D combinatorial/lattice factors. A particularly
important family of factors depends on a combined index $K$ (a sum of intermediate 1D indices), and
behaves like a discrete Hankel kernel. The code packages this dependence into tables
\begin{align}
  F_q[n] =
  \begin{cases}
    0, & n\ \text{odd},\\
    (-1)^{n/2}\, t_{n/2}\, s_q^n, & n\ \text{even},
  \end{cases}
\end{align}
with
\begin{align}
  t_n = \frac{\sqrt{(2n)!}}{2^n n!}.
\end{align}
The practical takeaway: rather than recomputing factorial-heavy expressions inside the RHS, we
precompute \code{Fq[q,n]} once and reuse it for every RHS evaluation. The truncation parameter
\code{maxK} caps the maximum $n$ needed; in practice \code{maxK} should comfortably exceed a few
multiples of $n_{\max}$ (the code uses a conservative default of \code{256}).

\subsection{MPO viewpoint: tensor-product operator contraction}
After SOE separation, the operator can be evaluated through a sequence of \emph{1D} mode products along
the $x$, $y$, and $z$ axes. In quantum many-body language this is a matrix-product operator (MPO); in
numerical linear algebra it is a Kronecker-structured operator.

\paragraph{A simple 2D analogy.}
Suppose $X\in\RR^{p\times p}$ and we want to apply a separable linear map
$Y = A\,X\,B^\top$ with $A,B\in\RR^{p\times p}$. In index notation,
\begin{align}
  Y_{ij} = \sum_{m=0}^{p-1}\sum_{n=0}^{p-1} A_{i m}\,X_{m n}\,B_{j n},
\end{align}
which looks like a dense $p^4$ contraction, but it can be implemented as two $p^3$ matrix multiplications:
first $Z=A X$, then $Y=Z B^\top$. This idea generalizes to 3D tensor products.

\paragraph{3D tensor-product action (mode products).}
For a 3D coefficient tensor $f\in\RR^{p\times p\times p}$ and three 1D matrices
$M^{(x)},M^{(y)},M^{(z)}\in\RR^{p'\times p}$, a separable contraction takes the form
\begin{align}
  (Mf)_{a b c} = \sum_{\alpha,\beta,\gamma}
  M^{(x)}_{a\alpha}\,M^{(y)}_{b\beta}\,M^{(z)}_{c\gamma}\;f_{\alpha\beta\gamma}.
\end{align}
Rather than forming the 3D operator explicitly, we compute this by three successive 1D contractions:
\begin{align}
  f^{(1)}_{a\beta\gamma} &= \sum_{\alpha} M^{(x)}_{a\alpha}\,f_{\alpha\beta\gamma},\\
  f^{(2)}_{a b\gamma} &= \sum_{\beta} M^{(y)}_{b\beta}\,f^{(1)}_{a\beta\gamma},\\
  (Mf)_{a b c} &= \sum_{\gamma} M^{(z)}_{c\gamma}\,f^{(2)}_{a b\gamma}.
\end{align}
Each step is a batched matrix multiplication. In the JAX backend, these are written as
\code{einsum}/\code{tensordot} and compiled/fused by XLA. The cost scales like a small constant times
$p'^{}p^3$ per term, rather than the naive $p^6$--$p^7$ dense couplings.

Concretely, the code precomputes (for each quadrature node $q$, tensor component indices $i,j$, and
term type) three 1D matrices that act along each dimension. Applying the operator to a 3D tensor then
reduces to three successive contractions (a Kronecker product action), implemented via
\code{einsum}/\code{tensordot} and fused by XLA in the JAX path.

\subsection{TT (tensor train) compression}
For larger $n_{\max}$, some intermediate tensors can become costly. A tensor-train (TT) factorization
approximates a 3D tensor $X\in\RR^{p\times p\times p}$ by low-rank factors (TT-SVD). The script includes
an optional TT rounding path (enabled via \code{--use\_tt} in the NumPy backend) to reduce intermediate
ranks and memory traffic when exploring higher $n_{\max}$.

\paragraph{What TT means (in one equation).}
A rank-$(r_1,r_2)$ TT approximation represents a 3D tensor as
\begin{align}
  X_{i j k} \approx \sum_{a=1}^{r_1}\sum_{b=1}^{r_2} G^{(1)}_{i a}\,G^{(2)}_{a j b}\,G^{(3)}_{b k},
\end{align}
where the cores $G^{(1)}\in\RR^{p\times r_1}$, $G^{(2)}\in\RR^{r_1\times p\times r_2}$, and
$G^{(3)}\in\RR^{r_2\times p}$ are found (approximately) by a sequence of SVDs (TT-SVD). If the ranks
stay small, storing and applying such tensors can be much cheaper than dense $p^3$ storage.

\paragraph{Why TT is optional here.}
On CPU, JAX/XLA excels at fusing the repeated dense contractions used in the MPO path. TT would require
SVDs, which are expensive and (for this project) not always a net win at moderate $n_{\max}$. Therefore
TT compression is currently used primarily for the NumPy debug path and for experimentation. The main
performance path is ``MPO + JIT''.

\paragraph{TT-SVD (how the factors are computed).}
The standard TT-SVD procedure for a 3D tensor $X\in\RR^{p\times p\times p}$ is:
\begin{enumerate}[nosep]
  \item Reshape $X$ into a matrix $X_{(1)}\in\RR^{p\times p^2}$ and compute its (truncated) SVD
  $X_{(1)}\approx U_1 \Sigma_1 V_1^\top$; set $G^{(1)}=U_1$ and absorb $\Sigma_1$ into $V_1^\top$.
  \item Reshape $\Sigma_1 V_1^\top$ into a matrix of shape $(r_1 p)\times p$ and compute another SVD
  to obtain $G^{(2)}$ and $G^{(3)}$.
\end{enumerate}
Truncation tolerances control the ranks $(r_1,r_2)$ and the approximation error. See
\cite{Oseledets2011,Hackbusch2012} for details and error bounds.

TT/MPS references: Oseledets (2011), Hackbusch (2012), and related MPO/MPS reviews (Schollw\"ock 2011; Or\'us 2014).

\subsection{Algorithmic walkthrough (as implemented)}
This subsection summarizes the concrete data flow in \code{landau\_hermite\_jax.py}.

\paragraph{Inputs and shapes.}
Fix a truncation $n_{\max}$ and set $p=n_{\max}+1$. Each distribution is a tensor
$f\in\RR^{p\times p\times p}$ in the Hermite basis.

\paragraph{Step 0: precompute tables (once per $(n_{\max},Q,\texttt{maxK})$ and species pair).}
The function \code{build\_model\_tables\_np} builds a container of dense arrays:
\begin{itemize}[nosep]
  \item Gauss--Legendre nodes/weights $(s_q,w_q)$ on $[0,1]$ for the separable quadrature.
  \item 1D mixing coefficients \code{a1d} encoding relative/COM transforms with the species-dependent
  angle $(\cos\theta,\sin\theta)$.
  \item A dense 1D coefficient table \code{P1D} that maps intermediate indices back to Hermite indices
  (a 1D ``dual$\leftrightarrow$primal'' transform used repeatedly in each dimension).
  \item The separable Hankel tables \code{Fq[q,K]} which implement the even-$K$ Coulomb moment factors
  multiplied by $s_q^K$ (see \code{Fq\_table\_np}).
  \item Preassembled 1D matrices \code{M\_buildS[q,i,j,term,dim]} used to build auxiliary tensors $S_1^{ij}$
  and $S_2^{ij}$ from the background distribution (next paragraph).
\end{itemize}
In the JAX backend these arrays are transferred to device once and treated as constants by JIT.

\paragraph{Step 1: build auxiliary tensors from $f_b$.}
For a cross-collision $Q_{ab}(f_a,f_b)$, the algorithm first constructs a small set of intermediate tensors
(denoted $S_1^{ij}$ and $S_2^{ij}$ in the code) that play the role of Rosenbluth-potential moments in this
Hermite formulation. Each $S_{\ell}^{ij}$ has shape $(p',p',p')$ where $p'=2n_{\max}+2$ (the intermediate
index range required by the Hankel structure). The construction is a sum over quadrature nodes $q$ of
Kronecker-structured applications:
\begin{align}
  S_{\ell}^{ij} \approx \sum_{q=1}^{Q} w_q\;\Big( M^{(x)}_{q,i,j,\ell}\otimes M^{(y)}_{q,i,j,\ell}\otimes M^{(z)}_{q,i,j,\ell} \Big)\, f_b,
\end{align}
implemented via three successive 1D contractions. This avoids forming any dense 6D objects.

\paragraph{Step 2: apply $S$ to $f_a$ to produce the RHS.}
Given $S_1,S_2$ (from $f_b$), the RHS for $f_a$ is assembled by applying another Kronecker-structured mapping
and then projecting back to the Hermite coefficient cube using the 1D \code{P1D} table along each axis.
This is the ``MPO'' viewpoint: the full operator factorizes over dimensions, so applying it is a sequence of
mode products rather than a giant tensor contraction.

\paragraph{Optional: TT compression (NumPy path).}
When \code{--use\_tt} is enabled in the NumPy backend, selected intermediate tensors may be TT-rounded to
reduce ranks. The JAX backend currently relies primarily on XLA fusion rather than TT-SVD, because SVD-based
compression is relatively expensive on CPU for the small-to-moderate $p$ used in the main figures.

\paragraph{Complexity.}
The dominant cost scales roughly like $O(Q\,p^4)$ for the structured contractions (with a small constant),
rather than $O(p^9)$ for a dense bilinear tensor.

\section{Nonlinear vs linearized evolution}

\subsection{Nonlinear operator}
The main ODE uses the full bilinear operator:
\begin{align}
  \text{1sp:}\quad \frac{\dd f}{\dd t} = Q_{11}(f,f),\qquad
  \text{2sp:}\quad \frac{\dd f_a}{\dd t} = Q_{ab}(f_a,f_b),\quad \frac{\dd f_b}{\dd t} = Q_{ba}(f_b,f_a).
\end{align}

\subsection{Linearization about a Maxwellian}
For many purposes it is useful to compare the full nonlinear evolution to a \emph{tangent-linear}
approximation. The script can optionally compute such ``linearized overlays'' with
\code{--linearized on}.

\paragraph{Which Maxwellian do we linearize about?}
The correct background is the \emph{equilibrium Maxwellian with the same conserved invariants} as the
initial condition.
For the 1-species case, this means the unique isotropic Maxwellian $M_{\mathrm{eq}}$ such that
\begin{align}
  (n,\bm P,W)[M_{\mathrm{eq}}]=(n,\bm P,W)[f_0].
\end{align}
For the 2-species cross-collision case, the equilibrium is a \emph{pair} $(M_{a,\mathrm{eq}},M_{b,\mathrm{eq}})$ with:
\begin{itemize}[nosep]
  \item each species keeps its own density ($n_a$ and $n_b$ are invariants),
  \item both share a common drift velocity set by total momentum,
  \item both share a common temperature set by total energy after removing drift energy.
\end{itemize}
In the code this is constructed by projecting the analytical Maxwellian onto the truncated Hermite basis
(see \code{build\_maxwellian\_tensor\_from\_invariants} and
\code{build\_common\_equilibrium\_maxwellians\_2sp\_from\_invariants}).

\paragraph{Why this matters (the ``plateau'' issue).}
The linearized Landau operator has a nontrivial nullspace corresponding to collision invariants. If one
linearizes about a Maxwellian $M$ that does \emph{not} match the IC invariants, then $h_0=f_0-M$ contains
nullspace components and they will not decay. In that case a quadratic free-energy diagnostic can
decrease but plateau above zero. The overlays in this script are therefore always constructed about the
invariant-matched equilibrium backgrounds.

\paragraph{Definition of the linearized operator (1 species).}
For a perturbation $h$ with $f=M_{\mathrm{eq}}+h$ and $\|h\|\ll 1$,
\begin{align}
  Q(M+h,M+h) = Q(M,M) + Q(h,M) + Q(M,h) + O(h^2).
\end{align}
Since $Q(M,M)=0$, the linearized operator is
\begin{align}
  L(h) = Q(h,M) + Q(M,h).
\end{align}
For two species, the linearized Jacobian couples $\delta f_a$ and $\delta f_b$ through the mixed terms.

\paragraph{How \code{--linearized\_method} works in the script.}
The CLI flag \code{--linearized\_method} only affects \emph{how we compute} the linearized time histories:
\begin{itemize}[nosep]
  \item \code{tangent} (default): matrix-free tangent-linear evolution using JAX directional derivatives
  (\code{jax.jvp}) in the JAX backend, or explicit bilinear expansions in the NumPy backend. This scales
  to larger $n_{\max}$ because it never forms a dense Jacobian.
  \item \code{matrix}: build an explicit dense Jacobian (via \code{jax.jacfwd} in JAX or a column-by-column
  fast assembly in NumPy) and then evolve with \code{expm\_multiply}. This is convenient for small $n_{\max}$
  but becomes expensive quickly.
\end{itemize}

\subsection{Entropy vs free energy for linearized dynamics}
This script uses two complementary Lyapunov-style diagnostics, each matched to the correct physics:

\paragraph{Nonlinear: KL divergence to the instantaneous local Maxwellian.}
The nonlinear Landau operator satisfies an H-theorem: a suitable relative entropy to the
\emph{instantaneous} local Maxwellian is non-increasing in time. The script measures this with a
grid-based Kullback--Leibler (KL) divergence
\begin{align}
  \mathcal{D}(t) = \int f \log\!\left(\frac{f}{M_{\mathrm{local}}}\right)\dd v.
\end{align}
Here $M_{\mathrm{local}}(\bm v,t)$ is the Maxwellian with the same $(n,\bm u,T)$ as $f(\bm v,t)$, i.e.
it depends on time through the low-order moments. For the spatially homogeneous setting, this ``local''
Maxwellian is global in space but local in moment space.

\paragraph{Explicit form of $M_{\mathrm{local}}$ (what the code evaluates).}
In physical variables,
\begin{align}
  M_{\mathrm{local}}(\bm v,t)=\frac{n(t)}{\left(2\pi T(t)/m\right)^{3/2}}
  \exp\!\left(-\frac{m|\bm v-\bm u(t)|^2}{2T(t)}\right),
\end{align}
with $n(t)=\int f\,\dd v$, $\bm u(t)=\bm P(t)/(m n(t))$, and $T(t)=\frac{2}{3}\frac{1}{n(t)}\int \frac{1}{2}m|\bm v-\bm u(t)|^2 f\,\dd v$.
In the code, all of these are computed from low-order Hermite coefficients (Sec.~5), then the integral
defining $\mathcal D$ is approximated on a fixed tensor-product grid in $(x,y,z)=\bm v/v_{\mathrm{th}}$.

\paragraph{Why this is a strong physics check.}
For the \emph{nonlinear} Landau operator, the H-theorem implies $\mathcal D(t)$ should be non-increasing
(up to discretization and truncation error). If $\mathcal D$ increases noticeably, it is a red flag:
either the operator is incorrect, or parameters $(n_{\max},Q,\texttt{maxK})$ are too low, or the entropy
grid is too coarse. This is why monotonicity of $\mathcal D$ is included in \code{--run\_tests}.

However, this KL functional is \emph{not} expected to be monotone for tangent-linear evolution (and can
become ill-defined if the linearized solution produces slight negative values due to truncation).

\paragraph{Linearized: quadratic free energy about the fixed equilibrium Maxwellian.}
For tangent-linear evolution about a fixed Maxwellian background, the natural monotone quantity is the
quadratic ``free energy'' (the second variation of entropy at $M$):
\begin{align}
  \mathcal{F}(t;M) = \int \frac{(f-M)^2}{M}\,\dd v.
\end{align}
In the script, $M$ is the invariant-matched equilibrium Maxwellian described above.
The script therefore plots $\mathcal{D}/\mathcal{D}_0$ for nonlinear curves and $\mathcal{F}/\mathcal{F}_0$
for linearized curves in the main panels, and it tests monotonicity for each in \code{--run\_tests}.

\section{Time stepping and JAX implementation}

\subsection{Integrator}
The script uses explicit SSPRK3 (Shu--Osher) by default, with options for RK2 and RK4. SSPRK3 offers a
good stability/accuracy compromise for dissipative problems while using only 3 RHS evaluations per step.
See Shu \& Osher (1988) and Gottlieb--Shu--Tadmor (2001).

\subsection{JAX performance strategy}
Key performance choices:
\begin{itemize}[nosep]
  \item Enable float64 (\code{jax\_enable\_x64}) for accurate invariants and Maxwellian fixed-point checks.
  \item Treat all precomputed tables as constants to JIT (single device transfer).
  \item Use \code{lax.scan} for the time loop so the full stepper is compiled once.
  \item Keep shapes static (fixed $p=n_{\max}+1$ and fixed $Q$ per run).
  \item Avoid Python loops over tensor indices; rely on vectorized contractions.
\end{itemize}
These are typical ``best practices'' for XLA-based array programming systems.

\section{Why JAX is a good fit for this solver (and how it reduces cost)}
\subsection{What is JAX (in practical terms)?}
JAX is a numerical computing library that looks like NumPy but is built around three ``program
transformations'':
\begin{itemize}[nosep]
  \item \textbf{automatic differentiation} (forward- and reverse-mode),
  \item \textbf{vectorization} (\code{vmap}: turn a function into a batched function),
  \item \textbf{compilation} (\code{jit}: compile array programs with XLA).
\end{itemize}
The key idea is that, instead of executing Python loops and NumPy calls directly, JAX can \emph{trace}
an array program and hand a static computation graph to the XLA compiler, which then generates optimized
machine code. That is the main reason this project benefits from JAX: the Landau--Hermite RHS is a large
composition of dense contractions with fixed shapes, repeated many times.

\subsection{Where JAX helps the most for this code}
At the level of the math, the fast RHS derived from Eq.~\eqref{eq:Cabkkakb_companion} boils down to:
\begin{itemize}[nosep]
  \item many \textbf{batched dense contractions} (mode products),
  \item repeated for many time steps,
  \item with \textbf{fixed shapes} once $n_{\max}$, $Q$, and \code{maxK} are chosen,
  \item and with large constant tables reused at every call.
\end{itemize}
This profile matches JAX/XLA extremely well: it can compile the entire computation graph ahead of time
and then execute fused kernels with minimal Python overhead. In contrast, pure Python/NumPy pays
per-call dispatch overhead and does not automatically fuse chains of contractions.

\subsection{Glossary: mode products, batched contractions, and static shapes}
The most common operation in this solver is a \emph{mode product}: contracting a tensor with a matrix
along one axis. For a 3D coefficient tensor $f\in\RR^{p\times p\times p}$ and a 1D operator
$A\in\RR^{p'\times p}$, the mode-$x$ product is
\begin{align}
  (f\times_x A)[i,j,k]=\sum_{\alpha=1}^{p} A[i,\alpha]\,f[\alpha,j,k].
\end{align}
Applying a Kronecker-product operator $(A_x\otimes A_y\otimes A_z)$ is then just three mode products
(first along $x$, then $y$, then $z$), as shown in Sec.~6.
In code, these appear as \code{einsum} calls such as:
\begin{Verbatim}[fontsize=\small]
# schematic: g = (Ax kron Ay kron Az) f
tmp = jnp.einsum("ia,ajk->ijk", Ax, f)      # x-mode product
tmp = jnp.einsum("jb,ibk->ijk", Ay, tmp)    # y-mode product
g   = jnp.einsum("kc,ijc->ijk", Az, tmp)    # z-mode product
\end{Verbatim}
``Static shapes'' means that the sizes $p$, $p'$, $Q$ are fixed for a run; this allows XLA to plan
memory and fuse kernels aggressively.

\subsection{How the script uses JAX transformations}
\paragraph{\code{jax.jit}: compile the RHS and the stepper.}
The JAX backend wraps the expensive kernels in \code{jax.jit}. Once compiled, subsequent calls run at
steady-state speed. This is why the test suite reports both compile time and per-call time.

\paragraph{\code{vmap}: batch over quadrature nodes.}
The SOE separation introduces a sum over quadrature nodes $q=1,\dots,Q$. In the JAX RHS, the work for
all $q$ is expressed as a batched computation using \code{vmap}, and then reduced by a weighted sum.
This removes Python-level loops over $q$ while keeping the math identical.

\paragraph{\code{lax.scan}: compile the time loop.}
Time integration is a perfect match for \code{lax.scan}: it turns ``for step in range(N)'' into a
single compiled loop. That reduces overhead and improves fusion opportunities.

\subsection{JAX building blocks used here}
\paragraph{Constants as compiled inputs.}
The script builds all coefficient tables once per run (depending on $n_{\max},Q,\code{maxK}$ and the
species pair). In the JAX path these arrays are transferred once and then treated as constants by the
compiled kernels. This avoids repeated allocations and helps XLA reuse memory.

\paragraph{Fusion and dispatch overhead.}
Many parts of the RHS are ``small'' elementwise operations surrounding large contractions. In NumPy
these would be multiple separate kernel launches and Python dispatches. XLA can fuse chains of
elementwise ops into the surrounding contraction kernels, reducing memory traffic and call overhead.

\subsection{How much does JAX reduce the cost? (measured in this repository)}
The \code{--run\_tests} suite measures both NumPy and JAX steady-state RHS times for small $n_{\max}$
values (NumPy comparisons are disabled at larger $n_{\max}$ because they become slow). The speedups are
summarized in Table~\ref{tab:tests_sweep} and in the dedicated speedup plot.

\begin{figure}[t]
  \centering
  \includegraphics[width=0.75\linewidth]{tests_landau_hermite/latest/speedup_numpy_vs_jax.png}
  \caption{Measured speedup (NumPy time / JAX time) for per-call RHS evaluation at the $n_{\max}$ where
  NumPy comparisons are enabled. The speedup comes primarily from JIT compilation and reduced Python
  overhead.}
  \label{fig:speedup}
\end{figure}

\begin{figure}[t]
  \centering
  \begin{subfigure}[t]{0.49\linewidth}
    \centering
    \includegraphics[width=\linewidth]{tests_landau_hermite/latest/runtime_rhs_scaling.png}
    \caption{RHS steady-state scaling.}
  \end{subfigure}\hfill
  \begin{subfigure}[t]{0.49\linewidth}
    \centering
    \includegraphics[width=\linewidth]{tests_landau_hermite/latest/runtime_integrate_short.png}
    \caption{Short integration benchmark.}
  \end{subfigure}
  \caption{Performance figures generated by \code{--run\_tests}. These plots are anchored to the exact
  kernels used in the main figures.}
  \label{fig:perf}
\end{figure}

\subsection{Why this stays differentiable}
All RHS computations in the JAX backend are written in terms of JAX primitives (array operations,
\code{einsum}, \code{lax} control flow). This means that, in principle, one can differentiate through
the RHS and through time stepping using \code{jax.grad} / \code{jax.jvp}/\code{jax.vjp}, subject to the
usual considerations (memory cost for reverse-mode through long time horizons, stiffness constraints,
etc.). The script is careful to avoid Python-side control flow inside jitted regions, which is a
common pitfall for differentiability.

\subsection{Citations and context}
For the JAX programming model and transformations, see the JAX reference \cite{JAX2018}. JAX relies on
the XLA compiler to generate optimized machine code for array programs; XLA is the compilation
technology that enables kernel fusion and loop lowering used implicitly by \code{jax.jit} and
\code{lax.scan} \cite{XLA}.

\section{Initial conditions used in Fig.\ 1 and Fig.\ 2}

\subsection{Fig.\ 1: one-species IC (two-stream-like, positivity-safe)}
The script supports two one-species IC families via \code{--fig1\_ic}:
\begin{itemize}[nosep]
  \item \code{twostream} (default): a manifestly nonnegative two-stream-like mixture designed to avoid negative
  slices at moderate $n_{\max}$.
  \item \code{prl\_m2}: the original low-order anisotropy IC (only second moments) used in the reference PRL panel logic.
\end{itemize}

\paragraph{Equilibrium Maxwellian in this basis.}
In the normalized coordinate $(x,y,z)=\bm v/v_{\mathrm{th}}$, the equilibrium Maxwellian used by this script is
simply the 3D tensor-product ground state
\begin{align}
  M_0(x,y,z)=\psi_0(x)\psi_0(y)\psi_0(z),
\end{align}
so a density-normalized Maxwellian is $f(x,y,z)=f_{000}\,M_0(x,y,z)$ with $f_{000}=n/v_{\mathrm{th}}^3$.

\paragraph{\code{twostream} (default): shifted-Maxwellian mixture projected to the truncated basis.}
The \emph{physical-space} target shape is the even mixture
\begin{align}
  g_u(x)=\tfrac12\Big(\psi_0(x-u)+\psi_0(x+u)\Big),
\end{align}
with $u=\code{--amp1}$ (stream separation in $v_x/v_{\mathrm{th}}$).
The 3D IC is then
\begin{align}
  f_{\mathrm{ts}}(x,y,z)=\left(\sum_{n=0}^{n_{\max}} c_n\,\psi_n(x)\right)\psi_0(y)\psi_0(z),
\end{align}
where the coefficients $c_n$ are obtained by a weighted $L^2$ projection of $g_u$ onto
$\mathrm{span}\{\psi_0,\dots,\psi_{n_{\max}}\}$ on a dense grid (see \code{build\_ic\_fig1\_1sp\_twostream}).
Finally, the coefficients are scaled so that the resulting tensor satisfies
\begin{align}
  f[0,0,0]=\frac{1}{v_{\mathrm{th}}^3}\quad\text{(density normalization used by this script)}.
\end{align}
Optionally (default on), the script enforces nonnegativity on diagnostic grids by writing
$f=f_M+\gamma(f-f_M)$ with $f_M$ the Maxwellian having the same $f_{000}$ and choosing the largest
$\gamma\in[0,1]$ such that the slice $f(v_x,0,0)$ and plane $f(v_x,v_y,0)$ remain nonnegative on the
diagnostic grids.

\paragraph{\code{prl\_m2}: low-order anisotropy IC (exact in coefficient space).}
When \code{--fig1\_ic prl\_m2} is selected, the IC is set directly in coefficient space:
\begin{align}
  f[0,0,0]&=\frac{1}{v_{\mathrm{th}}^3},\\
  f[2,0,0]&=2a,\qquad f[0,2,0]=-a,\qquad f[0,0,2]=-a,
\end{align}
with $a=\code{--amp1}$ and all other coefficients zero.

\subsection{Fig.\ 1: two-species IC (positivity-safe hot/equilibrium)}
The Fig.~1 two-species IC is designed to produce a clean temperature-exchange relaxation without requiring a
very cold Maxwellian to be represented at low $n_{\max}$ (which is a common source of Gibbs-like oscillations).
Let $M_{a,0}$ and $M_{b,0}$ denote the coefficient-space Maxwellians with $f_{000}=1/v_{\mathrm{th},s}^3$ for each
species $s\in\{a,b\}$ (i.e.\ only the $(0,0,0)$ coefficient is nonzero). The IC is:
\begin{itemize}[nosep]
  \item one species is left at its equilibrium Maxwellian,
  \item the other is ``heated'' by multiplying $M_{0}$ by a positive isotropic quadratic factor:
  \begin{align}
    f_{\mathrm{hot}}(x,y,z)=Z\,\psi_0(x)\psi_0(y)\psi_0(z)\,\Big(1+a_2(x^2+y^2+z^2)\Big),
  \end{align}
  and scaling $Z$ so that $f[0,0,0]=1/v_{\mathrm{th}}^3$.
\end{itemize}
The coefficient tensor for the polynomial factor is constructed exactly using the matrix representation of
multiplication by $x$ in the $\{\psi_n\}$ basis (see \code{\_poly\_ic\_tensor\_from\_coeffs}). The scalar $a_2\ge 0$
is then chosen by bisection so that the temperature diagnostic computed from the invariants matches the target
hot temperature $T_{\mathrm{hot}}=T_{\mathrm{eq}}+|\code{--dT2}|$.

\subsection{Fig.\ 2: far-from-Maxwellian ICs}
Fig.~2 uses the \emph{same dynamics and parameters} as Fig.~1, but with ICs that excite higher Hermite modes
(up to $n_{\max}=4$ by default) while remaining nonnegative on diagnostic grids.

\paragraph{1 species.}
The IC is constructed as an equilibrium Maxwellian times a positive even polynomial (up to quartic):
\begin{align}
  f_{\mathrm{Fig2}}(x,y,z)=Z\,\psi_0(x)\psi_0(y)\psi_0(z)\Big(1+a_{2x}x^2+a_{2y}y^2+a_{2z}z^2+a_{4x}x^4+a_{4y}y^4+a_{4z}z^4\Big),
\end{align}
with $Z$ chosen so that $f[0,0,0]=1/v_{\mathrm{th}}^3$. The coefficients are set exactly by the code as
\begin{align}
  \texttt{base2}=0.22\,s,\quad \texttt{base4}=0.10\,s\ \ (\text{if }n_{\max}\ge 4),\quad s=\code{--fig2\_strength},
\end{align}
and then
\begin{align}
  a_{2x}=\texttt{base2}(1+0.9\,a),\quad a_{2y}=a_{2z}=\texttt{base2}(1-0.4\,a),\qquad a=\code{--amp1},
\end{align}
with the same pattern for the quartic coefficients $a_{4x},a_{4y},a_{4z}$.

\paragraph{2 species.}
Species $A$ is made hotter (target $T_{\mathrm{eq}}+|\code{--dT2}|$) using an \emph{isotropic} polynomial with
quartic content (when $n_{\max}\ge 4$), tuning its quadratic coefficient by bisection.
Species $B$ is made non-Maxwellian but kept near $T_{\mathrm{eq}}$ by tuning its quadratic coefficient (which can
be negative) subject to a nonnegativity constraint on the diagnostic grids; if that tuning fails, the script
falls back to a pure Maxwellian for species $B$.

\section{Figures produced by the script}

\subsection{Main panels}
\begin{figure}[t]
  \centering
  \includegraphics[width=0.95\linewidth]{Fig1_panel.pdf}
  \caption{Fig.\ 1 panel produced by \code{landau\_hermite\_jax.py}.}
  \label{fig:fig1_panel}
\end{figure}

\begin{figure}[t]
  \centering
  \includegraphics[width=0.95\linewidth]{Fig2_panel.pdf}
  \caption{Fig.\ 2 panel produced by \code{landau\_hermite\_jax.py}.}
  \label{fig:fig2_panel}
\end{figure}

\subsection{Test suite summary and plots}
The \code{--run\_tests} mode runs a sweep over $n_{\max}$, checks Maxwellian fixed points, cross-invariant
rates, finite-difference consistency of the linearization (when feasible), and short-run performance. It
also performs physics-style checks of monotonicity (nonlinear KL entropy and linear quadratic free energy)
and generates convergence sweeps in $(Q,\texttt{maxK})$.

\paragraph{Interpreting the test outputs.}
The tests are designed around standard invariants and stability properties of the Landau operator:
\begin{itemize}[nosep]
  \item \textbf{Maxwellian fixed point}: $RHS(M)\approx 0$ is a strict algebraic property of the collision
  operator. In floating point, the residual should be near roundoff if the discretization/tables are correct.
  \item \textbf{Conservation}: for cross-collisions, $Q_{ab}$ and $Q_{ba}$ exchange momentum/energy but must
  conserve totals. The test plots $|d/dt\,(n_{\rm tot},P_{\rm tot},W_{\rm tot})|$ computed directly from the RHS.
  \item \textbf{H-theorem diagnostics}: nonlinear KL entropy to the instantaneous local Maxwellian should be
  non-increasing (up to truncation/time-discretization). Linearized dynamics should instead be monitored with
  quadratic free energy about the fixed Maxwellian background.
  \item \textbf{Linearized operator structure}: the linearized Landau operator about a Maxwellian is (formally)
  self-adjoint and dissipative in a Maxwellian-weighted inner product, once the nullspace (collision invariants)
  is removed; the test suite includes an approximate symmetry/dissipation check on a velocity grid
  \cite{Villani2002,HelanderSigmar}.
  \item \textbf{Maxwellians with varied densities}: multispecies equilibrium allows different densities; the
  test suite verifies $Q_{ab}(M_a,M_b)\approx 0$ when $M_a,M_b$ differ only by density factors.
  \item \textbf{Galilean invariance vs truncation}: a drifted Maxwellian requires infinitely many Hermite modes.
  The test suite therefore checks \emph{convergence} of the drifted-Maxwellian residual as $n_{\max}$ increases,
  rather than expecting exact zero at fixed truncation.
  \item \textbf{Linearization correctness}: two complementary checks are used:
    (i) finite-difference self-consistency (NumPy, small $n_{\max}$) and
    (ii) small-perturbation agreement between nonlinear and tangent-linear time traces (JAX).
  \item \textbf{Performance}: steady-state RHS timing (after compilation) and a short integration benchmark.
\end{itemize}

% Auto-generated from tests_landau_hermite/latest/summary.json
\begin{table}[t]
\centering
\caption{Test sweep summary (JAX backend). Reported times are steady-state. Conservation rates are for cross-collision totals $d/dt\,(n,P,W)$ evaluated on a random near-Maxwellian state.}
\label{tab:tests_sweep}
\begin{tabular}{r r r r r r r}
\toprule
$n_{\max}$ & $N=(n_{\max}{+}1)^3$ & $t_{\mathrm{self}}$ [ms] & $t_{\mathrm{cross}}$ [ms] & $t_{\mathrm{int}}$ [s] & $\|RHS(M)\|/\|M\|$ (cross) & $|dW_{\rm tot}/dt|$ \\
\midrule
2 & 27 & 0.093 & 0.177 & 0.014 & 4.23e-16 & 1.63e-15 \\
3 & 64 & 0.280 & 0.556 & 0.044 & 4.35e-16 & 8.50e-12 \\
4 & 125 & 0.584 & 1.141 & 0.092 & 5.98e-16 & 4.78e-11 \\
5 & 216 & 0.854 & 1.711 & 0.133 & 6.00e-16 & 3.26e-10 \\
\bottomrule
\end{tabular}
\end{table}


\begin{figure}[t]
  \centering
  \includegraphics[width=0.90\linewidth]{tests_landau_hermite/latest/runtime_rhs_scaling.png}
  \caption{Per-RHS-call runtime scaling across $n_{\max}$ (from the latest test run).}
  \label{fig:runtime_rhs_scaling}
\end{figure}

\begin{figure}[t]
  \centering
  \includegraphics[width=0.90\linewidth]{tests_landau_hermite/latest/maxwellian_residual.png}
  \caption{Maxwellian fixed-point residual $\|RHS(M)\|/\|M\|$ across $n_{\max}$.}
  \label{fig:maxw_res}
\end{figure}

\begin{figure}[t]
  \centering
  \includegraphics[width=0.90\linewidth]{tests_landau_hermite/latest/conservation_rates.png}
  \caption{Cross-collision conservation-rate check: total invariant rates computed from the RHS on a random near-Maxwellian state.}
  \label{fig:cons_rates}
\end{figure}

\begin{figure}[t]
  \centering
  \includegraphics[width=0.90\linewidth]{tests_landau_hermite/latest/physics_entropy_free_energy.png}
  \caption{Physics-style monotonicity check: nonlinear KL entropy to instantaneous local Maxwellians (solid) and linearized quadratic free energy about fixed Maxwellians (dashed).}
  \label{fig:entropy_free_energy}
\end{figure}

\begin{figure}[t]
  \centering
  \includegraphics[width=0.90\linewidth]{tests_landau_hermite/latest/rhs_backend_consistency.png}
  \caption{Backend consistency between NumPy and JAX for small $n_{\max}$ where NumPy comparisons are enabled.}
  \label{fig:backend_consistency}
\end{figure}

\begin{figure}[t]
  \centering
  \includegraphics[width=0.90\linewidth]{tests_landau_hermite/latest/linearization_fd_best.png}
  \caption{Finite-difference self-consistency check of the linearization (NumPy, where enabled).}
  \label{fig:lin_fd}
\end{figure}

\begin{figure}[t]
  \centering
  \includegraphics[width=0.90\linewidth]{tests_landau_hermite/latest/linearized_backend_consistency.png}
  \caption{Linearized operator consistency between NumPy (explicit linearization) and JAX (JVP-based linearization), where enabled.}
  \label{fig:lin_backend}
\end{figure}

\begin{figure}[t]
  \centering
  \includegraphics[width=0.90\linewidth]{tests_landau_hermite/latest/linearization_small_perturb_1sp.png}
  \caption{Small-perturbation check: nonlinear vs tangent-linear for the 1-species problem.}
  \label{fig:lin_small_1sp}
\end{figure}

\begin{figure}[t]
  \centering
  \includegraphics[width=0.90\linewidth]{tests_landau_hermite/latest/linearization_small_perturb_2sp.png}
  \caption{Small-perturbation check: nonlinear vs tangent-linear for the 2-species temperature-exchange metric.}
  \label{fig:lin_small_2sp}
\end{figure}

\begin{figure}[t]
  \centering
  \includegraphics[width=0.90\linewidth]{tests_landau_hermite/latest/jax_compile_scaling.png}
  \caption{JAX compile-time scaling across $n_{\max}$ (first call).}
  \label{fig:jax_compile}
\end{figure}

\begin{figure}[t]
  \centering
  \includegraphics[width=0.90\linewidth]{tests_landau_hermite/latest/runtime_integrate_short.png}
  \caption{Short integration benchmark (NumPy vs JAX where enabled).}
  \label{fig:integrate_short}
\end{figure}

\begin{figure}[t]
  \centering
  \includegraphics[width=0.90\linewidth]{tests_landau_hermite/latest/drifted_maxwellian_convergence.png}
  \caption{Drifted Maxwellian residual versus truncation: at fixed small drift $u$, the residual decreases as $n_{\max}$ increases. This is a practical proxy for Galilean invariance under spectral truncation.}
  \label{fig:drift_convergence}
\end{figure}

\begin{figure}[t]
  \centering
  \includegraphics[width=0.72\linewidth]{tests_landau_hermite/latest/cross_equilibrium_residuals.png}
  \caption{Two-species equilibrium check with different densities: $Q_{ab}(M_a,M_b)\approx 0$ and $Q_{ba}(M_b,M_a)\approx 0$ for Maxwellians with shared temperature/velocity but different density prefactors.}
  \label{fig:cross_density_equilibrium}
\end{figure}

\begin{figure}[t]
  \centering
  \includegraphics[width=0.90\linewidth]{tests_landau_hermite/latest/linearized_symmetry_dissipation.png}
  \caption{Linearized operator structure check (approximate, grid-based): symmetry defect in the weighted inner product and weighted dissipation values $\langle h, Lh\rangle_M$ for two random perturbations with collision-invariant components removed.}
  \label{fig:lin_sym_diss}
\end{figure}

\clearpage
\section{Interpreting results and limitations}

\subsection{Positivity}
Hermite spectral truncations do \emph{not} guarantee pointwise positivity of the reconstructed distribution
function, even if the underlying physical solution is nonnegative. This is a well-known spectral/Gibbs
phenomenon. To keep the plotted slices physically meaningful, the default Fig.~1 one-species IC is built
from a manifestly nonnegative two-stream mixture and is optionally ``damped'' in high modes to remain
nonnegative on diagnostic grids.

\subsection{Monotonicity}
For the nonlinear operator, $\mathcal{D}(t)$ (KL divergence to the instantaneous local Maxwellian) is the
referee-proof diagnostic. For the linearized evolution, the correct monotone quantity is the quadratic
free energy $\mathcal{F}(t;M)$ about the fixed Maxwellian background, not KL to a time-dependent Maxwellian.

\subsection{Truncation and parameter choices}
The separation accuracy is controlled by:
\begin{itemize}[nosep]
  \item $n_{\max}$: Hermite resolution (spectral truncation),
  \item $Q$: quadrature nodes in the SOE separation,
  \item \code{maxK}: maximum Hankel index in the Coulomb moment tables.
\end{itemize}
The test suite includes Q/\code{maxK} sweeps and Maxwellian fixed-point residuals to help select safe
defaults.

\paragraph{Practical note on $Q$ at larger $n_{\max}$.}
For moderate truncations ($n_{\max}\le 5$), $Q=8$ is typically sufficient to achieve Maxwellian
fixed-point residuals near roundoff in double precision. As $n_{\max}$ increases, however, the
quadrature error can become visible unless $Q$ is increased. For this reason, the script contains a
simple heuristic auto-selection for $Q$ at larger $n_{\max}$ (enabled by default and disable-able via
\code{--no\_auto\_Q}). The \code{--run\_tests} suite also includes a $Q$-sweep plot to make this
dependence explicit.

\subsection{Other fast Landau algorithms in the literature (future directions)}
The present script focuses on a Hermite basis and a separated (SOE) evaluation that is well matched to
tensor-product contractions. There are also alternative fast strategies for the Landau operator:
\begin{itemize}[nosep]
  \item \textbf{FFT-based fast spectral methods:} represent $f$ on a uniform velocity grid and evaluate
  the collision operator using Fourier techniques to reduce the cost of the nonlocal convolution-like
  structures (e.g. Pareschi--Russo--Toscani \cite{PareschiRussoToscani2000} and follow-ups). These
  methods can achieve $\mathcal{O}(N\log N)$-type costs in grid settings.
  \item \textbf{Hermite with local basis adaptation:} recenter/rescale the Hermite basis using local
  Maxwellian parameters (velocity shift and temperature) to reduce the number of modes needed to
  represent far-from-equilibrium states (e.g. the local-approximation Hermite strategy in
  \cite{HermiteLocalLandau2021}). This is a promising way to push to higher effective resolution
  without increasing global $n_{\max}$ as much.
  \item \textbf{Potential-based solvers:} compute Rosenbluth potentials via Poisson solves in velocity
  space, then apply drift/diffusion operators. In Fourier bases, Poisson solves are diagonal; in
  Hermite bases, they are structured but not diagonal, suggesting specialized solvers might help.
\end{itemize}
These directions are not implemented in this standalone script (which intentionally stays within one
file and one numerical framework), but the test suite and modular structure are designed to make
comparisons feasible.

\section{References (selected)}
\begin{thebibliography}{99}

\bibitem{Landau1936}
L.~D.~Landau, ``Kinetic equation for the Coulomb effect,'' \emph{Physikalische Zeitschrift der Sowjetunion} (1936).

\bibitem{Rosenbluth1957}
M.~N.~Rosenbluth, W.~M.~MacDonald, and D.~L.~Judd, ``Fokker--Planck equation for an inverse-square force,''
\emph{Physical Review} \textbf{107} (1957).

\bibitem{HelanderSigmar}
P.~Helander and D.~J.~Sigmar, \emph{Collisional Transport in Magnetized Plasmas}, Cambridge Univ. Press (2002).

\bibitem{Villani2002}
C.~Villani, ``A review of mathematical topics in collisional kinetic theory,'' in \emph{Handbook of Mathematical Fluid Dynamics}, Vol.~I (2002).

\bibitem{ShuOsher1988}
C.-W.~Shu and S.~Osher, ``Efficient implementation of essentially non-oscillatory shock-capturing schemes,''
\emph{Journal of Computational Physics} \textbf{77} (1988).

\bibitem{GottliebShuTadmor2001}
S.~Gottlieb, C.-W.~Shu, and E.~Tadmor, ``Strong stability-preserving high-order time discretization methods,''
\emph{SIAM Review} \textbf{43} (2001).

\bibitem{Oseledets2011}
I.~V.~Oseledets, ``Tensor-train decomposition,'' \emph{SIAM Journal on Scientific Computing} \textbf{33} (2011).

\bibitem{Hackbusch2012}
W.~Hackbusch, \emph{Tensor Spaces and Numerical Tensor Calculus}, Springer (2012).

\bibitem{Schollwoeck2011}
U.~Schollw\"ock, ``The density-matrix renormalization group in the age of matrix product states,''
\emph{Annals of Physics} \textbf{326} (2011).

\bibitem{Orus2014}
R.~Or\'us, ``A practical introduction to tensor networks: Matrix product states and projected entangled pair states,''
\emph{Annals of Physics} \textbf{349} (2014).

\bibitem{GolubWelsch}
G.~H.~Golub and J.~H.~Welsch, ``Calculation of Gauss quadrature rules,'' \emph{Mathematics of Computation} \textbf{23} (1969).

\bibitem{Boyd}
J.~P.~Boyd, \emph{Chebyshev and Fourier Spectral Methods}, 2nd ed., Dover (2001). (Background on spectral truncation/oscillations.)

\bibitem{Grad}
H.~Grad, ``On the kinetic theory of rarefied gases,'' \emph{Communications on Pure and Applied Mathematics} \textbf{2} (1949).

\bibitem{PareschiRussoToscani2000}
L.~Pareschi, G.~Russo, and G.~Toscani, ``Fast spectral methods for the Fokker--Planck--Landau collision operator,'' \emph{Journal of Computational Physics} \textbf{165} (2000).

\bibitem{HermiteLocalLandau2021}
R.~Li, Y.~Ren, and Y.~Wang, ``A Hermite spectral method for the spatially homogeneous Landau equation based on local approximation,'' \emph{Journal of Computational Physics} \textbf{434} (2021), 110235. (See also arXiv:2004.06484.)

\bibitem{BeylkinMonzon}
G.~Beylkin and L.~Monz\'on, ``Approximation by exponential sums revisited,'' \emph{Applied and Computational Harmonic Analysis} (2005).

\bibitem{HackbuschKhoromskij2006I}
W.~Hackbusch and B.~N.~Khoromskij, ``Low-rank Kronecker-product approximation to multidimensional nonlocal operators. Part I. Separable approximation of multi-variate functions,'' \emph{Computing} \textbf{76} (2006).

\bibitem{HackbuschKhoromskij2006II}
W.~Hackbusch and B.~N.~Khoromskij, ``Low-rank Kronecker-product approximation to multidimensional nonlocal operators. Part II. HKT representation of certain operators,'' \emph{Computing} \textbf{76} (2006).

\bibitem{ParrishHohensteinMartinez2012LS_THC}
R.~M.~Parrish, E.~G.~Hohenstein, T.~J.~Mart\'inez, and C.~D.~Sherrill, ``Tensor hypercontraction. I. Least-squares tensor hypercontraction for the electron repulsion integral tensor,'' \emph{J. Chem. Phys.} \textbf{137} (2012).

\bibitem{JAX2018}
J.~Bradbury et al., ``JAX: composable transformations of Python+NumPy programs,'' (2018). \url{https://github.com/jax-ml/jax}

\bibitem{XLA}
Google, ``XLA: Accelerated Linear Algebra,'' documentation/reference for the XLA optimizing compiler used by JAX/TF. \url{https://www.tensorflow.org/xla}

\end{thebibliography}

\section{Appendix: full CLI help}\label{app:cli}
For completeness, this is the exact output of \code{python landau\_hermite\_jax.py -h} captured at the
time this companion was generated:
\VerbatimInput[fontsize=\small]{landau_hermite_jax_help.txt}

\end{document}
