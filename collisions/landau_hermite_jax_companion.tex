% Companion documentation for landau_hermite_jax.py
% This file is intended to be self-contained and pedagogic. It is not a journal manuscript.
\documentclass[11pt]{article}

\usepackage[margin=1in]{geometry}
\usepackage{microtype}
\usepackage{amsmath,amssymb,mathtools}
\usepackage{bm}
\usepackage{siunitx}
\usepackage{booktabs}
\usepackage{graphicx}
\usepackage{subcaption}
\usepackage{xcolor}
\usepackage{hyperref}
\usepackage{enumitem}
\usepackage{fancyvrb}

\hypersetup{
  colorlinks=true,
  linkcolor=blue!55!black,
  urlcolor=blue!55!black,
  citecolor=blue!55!black
}

\graphicspath{{./}{tests_landau_hermite/}{tests_landau_hermite/latest/}}

\title{Companion Notes for \texttt{landau\_hermite\_jax.py}:\\
Fast Landau--Hermite Collisions via SOE$\to$MPO/TT (JAX-first)}
\author{(auto-generated companion for the standalone script)}
\date{\today}

\newcommand{\code}[1]{\texttt{#1}}
\newcommand{\dd}{\mathrm{d}}
\newcommand{\RR}{\mathbb{R}}

\begin{document}
\maketitle

\begin{abstract}
This document explains the numerics, implementation, and test suite used by the standalone script
\code{landau\_hermite\_jax.py}. The script implements a fast evaluation of the spatially homogeneous
Landau (Fokker--Planck) collision operator in a 3D tensor-product Hermite basis, using a separable
quadrature (``SOE'' style separation), tensor-product operator factorizations (``MPO'' viewpoint),
and optional tensor-train (TT) compression (currently used in the NumPy debug path; the JAX path
relies on JIT+vectorized contractions).

The goal is pedagogic completeness: what the code computes, how it is normalized, how it is
time-stepped, what ``linearized'' means in this context, why the tests are designed the way they are,
and how to interpret the produced figures (Fig.~1/Fig.~2 panels and the test plots).
\end{abstract}

\tableofcontents
\newpage

\section{Overview: equations, goals, and where the cost lives}

This companion is meant to read like a guided tour from first principles to the specific fast
implementation in \code{landau\_hermite\_jax.py}. The structure is:
\begin{enumerate}[nosep]
  \item What equation we solve and what is measured (physics).
  \item How we discretize velocity with a Hermite basis (numerics).
  \item Why the nonlinear Landau operator is expensive (bottlenecks).
  \item How separability (SOE), tensor-product contractions (MPO), and optional compression (TT) remove
  the bottlenecks.
  \item How the script time-steps, tests itself, and produces Fig.~1/Fig.~2 panels.
\end{enumerate}

\subsection{The equations solved by the script}
The Landau collision operator describes collisional relaxation in velocity space. In this project we
consider the \emph{spatially homogeneous} setting, where the distribution functions depend only on
velocity $\bm{v}\in\RR^3$ and time $t$.

\paragraph{One species.}
For a single species with distribution $f(\bm{v},t)$ we solve the ODE/PDE
\begin{align}
  \partial_t f = \nu\,Q(f,f),
\end{align}
where $\nu$ is a collision frequency (a scalar in this script), and $Q$ is the nonlinear Landau
collision operator.

\paragraph{Two species.}
For two species $a$ and $b$ we solve
\begin{align}
  \partial_t f_a &= \nu_{ab}\,Q_{ab}(f_a,f_b),\\
  \partial_t f_b &= \nu_{ba}\,Q_{ba}(f_b,f_a),
\end{align}
with parameters chosen so that the \emph{total} invariants (summed across species) are conserved.

\subsection{What we are trying to compute}
The script does \emph{not} discretize physical space; it only evolves velocity-space dynamics.
Concretely, it:
\begin{itemize}[nosep]
  \item represents each $f(\bm{v},t)$ in a tensor-product Hermite basis in the normalized velocity
  variables $(x,y,z)=(v_x/v_{\mathrm{th}},v_y/v_{\mathrm{th}},v_z/v_{\mathrm{th}})$;
  \item evolves the corresponding coefficient tensor in time using an explicit ODE integrator;
  \item computes diagnostics (density, momentum, energy, temperatures, anisotropy) and produces the
  Fig.~1/Fig.~2 panels;
  \item optionally (\code{--run\_tests}) runs a correctness/performance suite that produces additional
  plots and timing data.
\end{itemize}

\subsection{Why this is hard: the nonlinear Landau bottleneck}
At a high level, the nonlinear Landau operator is expensive for two reasons:
\begin{enumerate}[nosep]
  \item \textbf{Nonlocality in velocity:} $Q(f,f)$ involves an integral over $\bm{v}'$ with a
  long-range Coulomb kernel. Naively, evaluating this for many velocity degrees of freedom is
  comparable to a dense integral operator.
  \item \textbf{Mode coupling:} in a spectral representation, each output coefficient couples to many
  input coefficients. A naive coefficient-space implementation resembles a high-rank tensor
  contraction with $\mathcal{O}(p^6)$--$\mathcal{O}(p^7)$ work (depending on formulation), where
  $p=n_{\max}+1$ per velocity dimension.
\end{enumerate}
The core purpose of the ``SOE$\to$MPO/TT'' machinery is to replace these naive dense couplings by a sum
of \emph{separable} tensor-product contractions, so the cost is dominated by repeated 1D contractions
that are friendly to JIT compilation and cache/accelerator hardware.

\subsection{Roadmap of the fast algorithm (one paragraph)}
In the Hermite formulation used here, the coefficient formulas can be reorganized so that many factors
depend only on \emph{one-dimensional} indices in $x$, $y$, and $z$, while the Coulomb kernel moments
enter through a scalar factor that can be written as an integral over $s\in[0,1]$. Approximating that
scalar factor by Gauss--Legendre quadrature yields a \emph{finite sum of separable terms} (called
``SOE'' in this code base). Each term becomes a tensor-product operator acting along $x$, $y$, and $z$,
which can be applied by three successive 1D contractions (an ``MPO'' viewpoint). For larger
$n_{\max}$, some intermediate tensors can optionally be compressed using a tensor-train (TT) format
(currently in the NumPy debug path).

\section{How to run the code (quickstart and reproducibility)}

\subsection{What the script produces}
Running \code{python landau\_hermite\_jax.py} (defaults) writes:
\begin{itemize}[nosep]
  \item \code{Fig1\_panel.pdf} and \code{Fig1\_panel.png}: a 2$\times$2 diagnostic panel for a
  one-species relaxation problem and a two-species equilibration problem.
  \item \code{Fig2\_panel.pdf} and \code{Fig2\_panel.png}: same dynamics/parameters as Fig.~1, but
  with a different, strongly non-Maxwellian initial condition (IC) designed to remain nonnegative on
  diagnostic grids.
  \item Optionally (if \code{--run\_tests}), a timestamped directory under
  \code{tests\_landau\_hermite/run\_YYYYmmdd\_HHMMSS/} containing plots, CSV, and JSON summaries. The
  latest run is also linked as \code{tests\_landau\_hermite/latest}.
\end{itemize}

\subsection{Backend}
The script has two backends:
\begin{itemize}[nosep]
  \item \textbf{JAX backend} (\code{--backend jax}, default): uses \code{jax.jit}, \code{vmap}/\code{einsum},
  and \code{lax.scan} time stepping for performance and scalability.
  \item \textbf{NumPy backend} (\code{--backend numpy}): for debugging and for certain explicit checks.
\end{itemize}
Both backends use the \emph{same} precomputed coefficient tables and the same fast SOE$\to$MPO contraction
structure; the difference is whether contractions are executed with NumPy or JAX/XLA.

\subsection{CLI options}
The script self-documents its CLI via \code{-h}. For day-to-day use, the most important options are:
\begin{itemize}[nosep]
  \item \code{--nmax}: Hermite truncation ($p=n_{\max}+1$ modes per dimension).
  \item \code{--backend \{jax,numpy\}}: performance vs debugging.
  \item \code{--Q}, \code{--maxK}: accuracy/cost knobs for the separated kernel evaluation.
  \item \code{--linearized on/off}: include linearized overlays (computed within the same script).
  \item \code{--run\_tests}: generate the test suite plots and timing summaries in \code{tests\_landau\_hermite/}.
\end{itemize}
The full help text is included in Appendix~\ref{app:cli}.

\section{Physical model: Landau collision operator}

\subsection{Continuous operator (context)}
For species $a$ colliding with species $b$, the Landau collision operator can be written
(in one common form) as
\begin{align}
  Q_{ab}(f_a,f_b)(\bm{v})
  = \nabla_{\bm{v}}\cdot\!\int_{\RR^3} \mathbf{U}(\bm{v}-\bm{v}')
  \left[f_b(\bm{v}')\nabla_{\bm{v}} f_a(\bm{v}) - f_a(\bm{v})\nabla_{\bm{v}'} f_b(\bm{v}')\right]\,\dd \bm{v}',
\end{align}
where $\mathbf{U}$ is the Landau tensor kernel (Coulomb interaction case). This operator is:
\begin{itemize}[nosep]
  \item \textbf{bilinear} in $(f_a,f_b)$,
  \item \textbf{conservative}: it preserves total number, total momentum, and total energy
  (appropriate combinations across species),
  \item \textbf{entropy producing}: for the \emph{nonlinear} operator, relative entropy to a local Maxwellian
  is non-increasing in time (H-theorem).
\end{itemize}
Standard references include Landau (1936), Rosenbluth--MacDonald--Judd (1957), and modern expositions such
as Helander \& Sigmar (2002) and Villani (2002).

\subsection{What the script discretizes}
The code discretizes \emph{velocity} in a tensor-product Hermite basis and evolves the corresponding
expansion coefficients in time for spatially homogeneous relaxation. All spatial dependence is absent:
the ODE is in coefficient space.

\section{Normalization, basis, and coefficient tensor representation}

\subsection{Thermal normalization}
Each species has a reference thermal speed $v_{\mathrm{th}}$ and mass $m$. The code uses normalized
velocity coordinates
\begin{align}
  x = v_x / v_{\mathrm{th}},\qquad y = v_y / v_{\mathrm{th}},\qquad z = v_z / v_{\mathrm{th}}.
\end{align}
The convention used throughout is
\begin{align}
  v_{\mathrm{th}} = \sqrt{\frac{2T_{\mathrm{eq}}}{m}}
  \quad \Longleftrightarrow \quad
  T_{\mathrm{eq}} = \frac{m v_{\mathrm{th}}^2}{2},
\end{align}
where $T_{\mathrm{eq}}$ is the equilibrium temperature corresponding to the chosen normalization.

\subsection{Hermite basis used in the code}
The script uses a Hermite--Gaussian basis in each dimension:
\begin{align}
  \psi_n(x) = \frac{1}{\sqrt{\pi}}\frac{H_n(x)}{\sqrt{2^n n!}}\,e^{-x^2},
\end{align}
where $H_n$ is the physicists' Hermite polynomial. The code implements $\psi_n$ in \code{psi\_1d}.

\paragraph{Why Hermite? (intuition)}
Hermite--Gaussian bases are well matched to collisional kinetic problems because:
\begin{itemize}[nosep]
  \item Maxwellians are Gaussians, so the equilibrium state is captured very compactly
  (in this script, the equilibrium Maxwellian corresponds to a single nonzero coefficient
  $f[0,0,0]$ for each species).
  \item Multiplication by $x$ and differentiation in $x$ act by \emph{ladder relations} that shift $n$
  by $\pm 1$, making many velocity-moment and derivative operations sparse in coefficient space.
  \item A tensor-product basis makes 3D operations separable across $(x,y,z)$, which is essential for
  the MPO-style contractions used later.
\end{itemize}

\paragraph{Orthogonality and normalization.}
With the above convention, $\{\psi_n\}_{n\ge 0}$ are orthonormal on $\RR$:
\begin{align}
  \int_{\RR} \psi_n(x)\,\psi_m(x)\,\dd x = \delta_{nm}.
\end{align}
The 3D tensor basis $\psi_{\alpha}(x)\psi_{\beta}(y)\psi_{\gamma}(z)$ is then orthonormal on $\RR^3$,
so inner products and many diagnostics reduce to sums over low-order coefficients.

\paragraph{Ladder identities (used implicitly).}
The Hermite functions satisfy identities of the schematic form
\begin{align}
  x\,\psi_n(x) &\propto \sqrt{n+1}\,\psi_{n+1}(x) + \sqrt{n}\,\psi_{n-1}(x),\\
  \partial_x \psi_n(x) &\propto \sqrt{n+1}\,\psi_{n+1}(x) - \sqrt{n}\,\psi_{n-1}(x),
\end{align}
up to fixed constants determined by the chosen normalization. In coefficient space, these become sparse
``shift'' operators. The fast Landau operator construction uses these structured shifts when building
the 1D tables.

\subsection{Tensor coefficient representation}
For a truncation parameter $n_{\max}$, define $p=n_{\max}+1$. The coefficient tensor is
\begin{align}
  f[\alpha,\beta,\gamma], \quad \alpha,\beta,\gamma\in\{0,\dots,n_{\max}\},
\end{align}
so that
\begin{align}
  f(x,y,z) \approx \sum_{\alpha,\beta,\gamma=0}^{n_{\max}} f[\alpha,\beta,\gamma]\,
  \psi_{\alpha}(x)\psi_{\beta}(y)\psi_{\gamma}(z).
\end{align}
The code keeps this 3D ``cube'' representation (shape $(p,p,p)$) throughout to avoid flattening overhead
and to make tensor contractions explicit and JIT-friendly.

\subsection{Moment diagnostics and invariants}
From the Hermite coefficients, the script extracts:
\begin{itemize}[nosep]
  \item density $n$,
  \item momentum $\bm{P}$,
  \item total kinetic energy $W$,
  \item temperature $T = \frac{2}{3}\frac{W}{n}$,
  \item anisotropy measure $A = (T_z-T_x)/T_{\mathrm{avg}}$.
\end{itemize}
These are computed from low-order coefficients (see \code{invariants\_from\_tensor} and
\code{temperature\_components\_hat\_from\_tensor}). Conservation of $(n,\bm{P},W)$ (or total invariants
for 2 species) is one of the most stringent correctness checks for collision operators.

\section{Fast collision evaluation: SOE$\to$MPO/TT}

\subsection{Why naive evaluation is expensive}
In a generic coefficient formulation, the bilinear operator has the form
\begin{align}
  \frac{\dd f_k}{\dd t} = \sum_{k_a,k_b} C^{ab}_{k,k_a,k_b}\,f_{a,k_a}\,f_{b,k_b},
\end{align}
which is $O(N^3)$ if the dense tensor $C^{ab}$ is formed and contracted directly (with $N=p^3$).
Even storing $C^{ab}$ is prohibitive.

The script avoids forming $C^{ab}$ and instead uses a structured evaluation that exploits separability
of the underlying integrals and the tensor-product basis.

\subsection{From dense mode coupling to a sum of separable contractions}
The phrase ``SOE$\to$MPO'' can feel like jumping into implementation details. Here is the intuitive
progression, starting from the equation.

\paragraph{Step 1: project the continuous operator onto a tensor-product basis.}
Because we use a tensor-product basis in $(x,y,z)$, the projection of $Q(f_a,f_b)$ onto a mode
$k=(\alpha,\beta,\gamma)$ naturally produces many factors that separate by dimension:
\begin{align}
  \langle \psi_{\alpha}\psi_{\beta}\psi_{\gamma},\; Q(f_a,f_b)\rangle
  \sim \sum_{\text{many indices}} \big(\text{1D factor in }x\big)\big(\text{1D factor in }y\big)\big(\text{1D factor in }z\big)\times \big(\text{kernel moment}\big).
\end{align}
The difficulty is that the \emph{kernel moment} couples the three dimensions through combined indices,
so at this stage a naive implementation still looks like a dense multi-index sum.

\paragraph{Step 2: identify the scalar ``kernel-moment'' factor that blocks separability.}
In the Hermite formulation used in this project, after reducing the Landau tensor kernel moments
(and performing the necessary combinatorics from Hermite multiplications/derivatives), the remaining
non-separable part is a scalar weight that depends on an integer index $A$ built from a \emph{sum}
of three 1D indices:
\begin{align}
  A = A_x + A_y + A_z,
\end{align}
and the coefficient formulas contain weights of the schematic form
\begin{align}
  \frac{1}{2A+1}\;\times\;(\text{products of 1D combinatorial factors}).
\end{align}
If we can rewrite $\frac{1}{2A+1}$ in a way that factorizes with respect to $A_x,A_y,A_z$, then the
entire operator becomes a sum of fully separable tensor-product contractions.

\subsection{SOE idea: rewrite the blocking scalar factor as an integral, then quadrature}
The key identity is not pulled from thin air: it is simply the Beta-function integral for a rational
factor:
\begin{align}
  \frac{1}{2A+1} = \int_{0}^{1} s^{2A}\,\dd s.
\end{align}
Now substitute $A=A_x+A_y+A_z$:
\begin{align}
  s^{2A} = s^{2A_x}\,s^{2A_y}\,s^{2A_z},
\end{align}
which is the crucial separation across dimensions. Finally, approximate the integral using
Gauss--Legendre quadrature on $[0,1]$:
\begin{align}
  \frac{1}{2A+1}
  \approx \sum_{q=1}^{Q} w_q\, s_q^{2A}
  = \sum_{q=1}^{Q} w_q\;\Big(s_q^{2A_x}\Big)\Big(s_q^{2A_y}\Big)\Big(s_q^{2A_z}\Big).
\end{align}
Each quadrature node $q$ produces a \emph{fully separable} contribution; the full operator is a sum of
$Q$ separable terms. This is what the code refers to as ``SOE'': a \textbf{sum of separable terms}
(also called a separated representation or sum-of-products). The quadrature size $Q$ is small
(default $Q=8$), so looping over $q$ is cheap and amenable to \code{vmap} or \code{lax.fori\_loop} in JAX.

\paragraph{Why Gauss--Legendre on $[0,1]$?}
Because the identity is an integral on $[0,1]$, Gauss--Legendre provides a high-order approximation
with very few nodes. This project uses the term ``SOE'' historically; the crucial feature is the
finite separated sum, not exponentials per se.

\subsection{Hankel-like tables: precomputing the remaining 1D dependence}
After SOE separation, what remains are purely 1D combinatorial/lattice factors. A particularly
important family of factors depends on a combined index $K$ (a sum of intermediate 1D indices), and
behaves like a discrete Hankel kernel. The code packages this dependence into tables
\begin{align}
  F_q[n] =
  \begin{cases}
    0, & n\ \text{odd},\\
    (-1)^{n/2}\, t_{n/2}\, s_q^n, & n\ \text{even},
  \end{cases}
\end{align}
with
\begin{align}
  t_n = \frac{\sqrt{(2n)!}}{2^n n!}.
\end{align}
The practical takeaway: rather than recomputing factorial-heavy expressions inside the RHS, we
precompute \code{Fq[q,n]} once and reuse it for every RHS evaluation. The truncation parameter
\code{maxK} caps the maximum $n$ needed; in practice \code{maxK} should comfortably exceed a few
multiples of $n_{\max}$ (the code uses a conservative default of \code{256}).

\subsection{MPO viewpoint: tensor-product operator contraction}
After SOE separation, the operator can be evaluated through a sequence of \emph{1D} mode products along
the $x$, $y$, and $z$ axes. In quantum many-body language this is a matrix-product operator (MPO); in
numerical linear algebra it is a Kronecker-structured operator.

\paragraph{A simple 2D analogy.}
Suppose $X\in\RR^{p\times p}$ and we want to apply a separable linear map
$Y = A\,X\,B^\top$ with $A,B\in\RR^{p\times p}$. In index notation,
\begin{align}
  Y_{ij} = \sum_{m=0}^{p-1}\sum_{n=0}^{p-1} A_{i m}\,X_{m n}\,B_{j n},
\end{align}
which looks like a dense $p^4$ contraction, but it can be implemented as two $p^3$ matrix multiplications:
first $Z=A X$, then $Y=Z B^\top$. This idea generalizes to 3D tensor products.

\paragraph{3D tensor-product action (mode products).}
For a 3D coefficient tensor $f\in\RR^{p\times p\times p}$ and three 1D matrices
$M^{(x)},M^{(y)},M^{(z)}\in\RR^{p'\times p}$, a separable contraction takes the form
\begin{align}
  (Mf)_{a b c} = \sum_{\alpha,\beta,\gamma}
  M^{(x)}_{a\alpha}\,M^{(y)}_{b\beta}\,M^{(z)}_{c\gamma}\;f_{\alpha\beta\gamma}.
\end{align}
Rather than forming the 3D operator explicitly, we compute this by three successive 1D contractions:
\begin{align}
  f^{(1)}_{a\beta\gamma} &= \sum_{\alpha} M^{(x)}_{a\alpha}\,f_{\alpha\beta\gamma},\\
  f^{(2)}_{a b\gamma} &= \sum_{\beta} M^{(y)}_{b\beta}\,f^{(1)}_{a\beta\gamma},\\
  (Mf)_{a b c} &= \sum_{\gamma} M^{(z)}_{c\gamma}\,f^{(2)}_{a b\gamma}.
\end{align}
Each step is a batched matrix multiplication. In the JAX backend, these are written as
\code{einsum}/\code{tensordot} and compiled/fused by XLA. The cost scales like a small constant times
$p'^{}p^3$ per term, rather than the naive $p^6$--$p^7$ dense couplings.

Concretely, the code precomputes (for each quadrature node $q$, tensor component indices $i,j$, and
term type) three 1D matrices that act along each dimension. Applying the operator to a 3D tensor then
reduces to three successive contractions (a Kronecker product action), implemented via
\code{einsum}/\code{tensordot} and fused by XLA in the JAX path.

\subsection{TT (tensor train) compression}
For larger $n_{\max}$, some intermediate tensors can become costly. A tensor-train (TT) factorization
approximates a 3D tensor $X\in\RR^{p\times p\times p}$ by low-rank factors (TT-SVD). The script includes
an optional TT rounding path (enabled via \code{--use\_tt} in the NumPy backend) to reduce intermediate
ranks and memory traffic when exploring higher $n_{\max}$.

\paragraph{What TT means (in one equation).}
A rank-$(r_1,r_2)$ TT approximation represents a 3D tensor as
\begin{align}
  X_{i j k} \approx \sum_{a=1}^{r_1}\sum_{b=1}^{r_2} G^{(1)}_{i a}\,G^{(2)}_{a j b}\,G^{(3)}_{b k},
\end{align}
where the cores $G^{(1)}\in\RR^{p\times r_1}$, $G^{(2)}\in\RR^{r_1\times p\times r_2}$, and
$G^{(3)}\in\RR^{r_2\times p}$ are found (approximately) by a sequence of SVDs (TT-SVD). If the ranks
stay small, storing and applying such tensors can be much cheaper than dense $p^3$ storage.

\paragraph{Why TT is optional here.}
On CPU, JAX/XLA excels at fusing the repeated dense contractions used in the MPO path. TT would require
SVDs, which are expensive and (for this project) not always a net win at moderate $n_{\max}$. Therefore
TT compression is currently used primarily for the NumPy debug path and for experimentation. The main
performance path is ``MPO + JIT''.

TT/MPS references: Oseledets (2011), Hackbusch (2012), and related MPO/MPS reviews (Schollw\"ock 2011; Or\'us 2014).

\subsection{Algorithmic walkthrough (as implemented)}
This subsection summarizes the concrete data flow in \code{landau\_hermite\_jax.py}.

\paragraph{Inputs and shapes.}
Fix a truncation $n_{\max}$ and set $p=n_{\max}+1$. Each distribution is a tensor
$f\in\RR^{p\times p\times p}$ in the Hermite basis.

\paragraph{Step 0: precompute tables (once per $(n_{\max},Q,\texttt{maxK})$ and species pair).}
The function \code{build\_model\_tables\_np} builds a container of dense arrays:
\begin{itemize}[nosep]
  \item Gauss--Legendre nodes/weights $(s_q,w_q)$ on $[0,1]$ for the separable quadrature.
  \item 1D mixing coefficients \code{a1d} encoding relative/COM transforms with the species-dependent
  angle $(\cos\theta,\sin\theta)$.
  \item A dense 1D coefficient table \code{P1D} that maps intermediate indices back to Hermite indices
  (a 1D ``dual$\leftrightarrow$primal'' transform used repeatedly in each dimension).
  \item The separable Hankel tables \code{Fq[q,K]} which implement the even-$K$ Coulomb moment factors
  multiplied by $s_q^K$ (see \code{Fq\_table\_np}).
  \item Preassembled 1D matrices \code{M\_buildS[q,i,j,term,dim]} used to build auxiliary tensors $S_1^{ij}$
  and $S_2^{ij}$ from the background distribution (next paragraph).
\end{itemize}
In the JAX backend these arrays are transferred to device once and treated as constants by JIT.

\paragraph{Step 1: build auxiliary tensors from $f_b$.}
For a cross-collision $Q_{ab}(f_a,f_b)$, the algorithm first constructs a small set of intermediate tensors
(denoted $S_1^{ij}$ and $S_2^{ij}$ in the code) that play the role of Rosenbluth-potential moments in this
Hermite formulation. Each $S_{\ell}^{ij}$ has shape $(p',p',p')$ where $p'=2n_{\max}+2$ (the intermediate
index range required by the Hankel structure). The construction is a sum over quadrature nodes $q$ of
Kronecker-structured applications:
\begin{align}
  S_{\ell}^{ij} \approx \sum_{q=1}^{Q} w_q\;\Big( M^{(x)}_{q,i,j,\ell}\otimes M^{(y)}_{q,i,j,\ell}\otimes M^{(z)}_{q,i,j,\ell} \Big)\, f_b,
\end{align}
implemented via three successive 1D contractions. This avoids forming any dense 6D objects.

\paragraph{Step 2: apply $S$ to $f_a$ to produce the RHS.}
Given $S_1,S_2$ (from $f_b$), the RHS for $f_a$ is assembled by applying another Kronecker-structured mapping
and then projecting back to the Hermite coefficient cube using the 1D \code{P1D} table along each axis.
This is the ``MPO'' viewpoint: the full operator factorizes over dimensions, so applying it is a sequence of
mode products rather than a giant tensor contraction.

\paragraph{Optional: TT compression (NumPy path).}
When \code{--use\_tt} is enabled in the NumPy backend, selected intermediate tensors may be TT-rounded to
reduce ranks. The JAX backend currently relies primarily on XLA fusion rather than TT-SVD, because SVD-based
compression is relatively expensive on CPU for the small-to-moderate $p$ used in the main figures.

\paragraph{Complexity.}
The dominant cost scales roughly like $O(Q\,p^4)$ for the structured contractions (with a small constant),
rather than $O(p^9)$ for a dense bilinear tensor.

\section{Nonlinear vs linearized evolution}

\subsection{Nonlinear operator}
The main ODE uses the full bilinear operator:
\begin{align}
  \text{1sp:}\quad \frac{\dd f}{\dd t} = Q_{11}(f,f),\qquad
  \text{2sp:}\quad \frac{\dd f_a}{\dd t} = Q_{ab}(f_a,f_b),\quad \frac{\dd f_b}{\dd t} = Q_{ba}(f_b,f_a).
\end{align}

\subsection{Linearization about a Maxwellian}
Let $M$ denote a Maxwellian equilibrium (in the \emph{fixed} normalized coordinates for each species).
For a perturbation $h$ with $f=M+h$ and $\|h\|\ll 1$,
\begin{align}
  Q(M+h,M+h) = Q(M,M) + Q(h,M) + Q(M,h) + O(h^2).
\end{align}
Since $Q(M,M)=0$, the linearized operator is
\begin{align}
  L(h) = Q(h,M) + Q(M,h).
\end{align}
For two species, the linearized Jacobian couples $\delta f_a$ and $\delta f_b$ through the mixed terms.

\subsection{Entropy vs free energy for linearized dynamics}
The nonlinear Landau operator satisfies an H-theorem: relative entropy to the \emph{instantaneous} local
Maxwellian decreases. The script measures this via a grid-based KL divergence:
\begin{align}
  \mathcal{D}(t) = \int f \log\!\left(\frac{f}{M_{\mathrm{local}}}\right)\dd v.
\end{align}
However, this KL functional is \emph{not} expected to be monotone for tangent-linear evolution (and can
become ill-defined if the linearized solution produces slight negative values due to truncation).

For linearized evolution, a standard Lyapunov functional is the quadratic ``free energy''
(second variation of entropy at a Maxwellian):
\begin{align}
  \mathcal{F}(t;M) = \int \frac{(f-M)^2}{M}\,\dd v.
\end{align}
The script therefore plots $\mathcal{D}/\mathcal{D}_0$ for nonlinear curves and $\mathcal{F}/\mathcal{F}_0$
for linearized curves in the main panels, and it tests monotonicity for each in \code{--run\_tests}.

\section{Time stepping and JAX implementation}

\subsection{Integrator}
The script uses explicit SSPRK3 (Shu--Osher) by default, with options for RK2 and RK4. SSPRK3 offers a
good stability/accuracy compromise for dissipative problems while using only 3 RHS evaluations per step.
See Shu \& Osher (1988) and Gottlieb--Shu--Tadmor (2001).

\subsection{JAX performance strategy}
Key performance choices:
\begin{itemize}[nosep]
  \item Enable float64 (\code{jax\_enable\_x64}) for accurate invariants and Maxwellian fixed-point checks.
  \item Treat all precomputed tables as constants to JIT (single device transfer).
  \item Use \code{lax.scan} for the time loop so the full stepper is compiled once.
  \item Keep shapes static (fixed $p=n_{\max}+1$ and fixed $Q$ per run).
  \item Avoid Python loops over tensor indices; rely on vectorized contractions.
\end{itemize}
These are typical ``best practices'' for XLA-based array programming systems.

\section{Initial conditions used in Fig.\ 1 and Fig.\ 2}

\subsection{Fig.\ 1: one-species IC (two-stream-like, positivity-safe)}
The default Fig.~1 one-species IC is designed to be qualitatively ``two-stream-like'' without any negative
values. It is constructed as an even mixture of two shifted Maxwellians in $v_x$:
\begin{align}
  f(x,y,z)\ \propto\ \frac{1}{2}\Big(M(x-u)+M(x+u)\Big)\,M(y)\,M(z),
\end{align}
with $u$ set by \code{--amp1} (interpreted as stream separation for \code{--fig1\_ic twostream}).
The code projects the 1D $x$ dependence onto the truncated $\{\psi_n\}$ basis and optionally scales the
non-Maxwellian component to enforce nonnegativity on diagnostic grids.

\subsection{Fig.\ 1: two-species IC (positivity-safe hot/equilibrium)}
Representing an extremely cold Maxwellian at low $n_{\max}$ in a fixed normalized coordinate can lead to
Gibbs-like oscillations and negative reconstructed $f$.
To avoid this, the two-species IC used in this standalone script heats one species using a positive
isotropic polynomial distortion of the equilibrium Maxwellian (exactly representable at low order), while
keeping the other species near equilibrium. The parameter \code{--dT2} sets the target temperature excess
for the hot species.

\subsection{Fig.\ 2: far-from-Maxwellian ICs}
Fig.~2 uses a positive polynomial distortion (including quartic terms when $n_{\max}\ge 4$) to create
strongly non-Maxwellian shapes while maintaining nonnegativity on diagnostic grids. The strength is
controlled by \code{--fig2\_strength}. The two-species Fig.~2 IC also tunes the temperature separation in
a positivity-safe way.

\section{Figures produced by the script}

\subsection{Main panels}
\begin{figure}[t]
  \centering
  \includegraphics[width=0.95\linewidth]{Fig1_panel.pdf}
  \caption{Fig.\ 1 panel produced by \code{landau\_hermite\_jax.py}.}
  \label{fig:fig1_panel}
\end{figure}

\begin{figure}[t]
  \centering
  \includegraphics[width=0.95\linewidth]{Fig2_panel.pdf}
  \caption{Fig.\ 2 panel produced by \code{landau\_hermite\_jax.py}.}
  \label{fig:fig2_panel}
\end{figure}

\subsection{Test suite summary and plots}
The \code{--run\_tests} mode runs a sweep over $n_{\max}$, checks Maxwellian fixed points, cross-invariant
rates, finite-difference consistency of the linearization (when feasible), and short-run performance. It
also performs physics-style checks of monotonicity (nonlinear KL entropy and linear quadratic free energy)
and generates convergence sweeps in $(Q,\texttt{maxK})$.

\paragraph{Interpreting the test outputs.}
The tests are designed around standard invariants and stability properties of the Landau operator:
\begin{itemize}[nosep]
  \item \textbf{Maxwellian fixed point}: $RHS(M)\approx 0$ is a strict algebraic property of the collision
  operator. In floating point, the residual should be near roundoff if the discretization/tables are correct.
  \item \textbf{Conservation}: for cross-collisions, $Q_{ab}$ and $Q_{ba}$ exchange momentum/energy but must
  conserve totals. The test plots $|d/dt\,(n_{\rm tot},P_{\rm tot},W_{\rm tot})|$ computed directly from the RHS.
  \item \textbf{H-theorem diagnostics}: nonlinear KL entropy to the instantaneous local Maxwellian should be
  non-increasing (up to truncation/time-discretization). Linearized dynamics should instead be monitored with
  quadratic free energy about the fixed Maxwellian background.
  \item \textbf{Linearization correctness}: two complementary checks are used:
    (i) finite-difference self-consistency (NumPy, small $n_{\max}$) and
    (ii) small-perturbation agreement between nonlinear and tangent-linear time traces (JAX).
  \item \textbf{Performance}: steady-state RHS timing (after compilation) and a short integration benchmark.
\end{itemize}

% Auto-generated from tests_landau_hermite/latest/summary.json
\begin{table}[t]
\centering
\caption{Test sweep summary (JAX backend). Reported times are steady-state. Conservation rates are for cross-collision totals $d/dt\,(n,P,W)$ evaluated on a random near-Maxwellian state.}
\label{tab:tests_sweep}
\begin{tabular}{r r r r r r r}
\toprule
$n_{\max}$ & $N=(n_{\max}{+}1)^3$ & $t_{\mathrm{self}}$ [ms] & $t_{\mathrm{cross}}$ [ms] & $t_{\mathrm{int}}$ [s] & $\|RHS(M)\|/\|M\|$ (cross) & $|dW_{\rm tot}/dt|$ \\
\midrule
2 & 27 & 0.093 & 0.177 & 0.014 & 4.23e-16 & 1.63e-15 \\
3 & 64 & 0.280 & 0.556 & 0.044 & 4.35e-16 & 8.50e-12 \\
4 & 125 & 0.584 & 1.141 & 0.092 & 5.98e-16 & 4.78e-11 \\
5 & 216 & 0.854 & 1.711 & 0.133 & 6.00e-16 & 3.26e-10 \\
\bottomrule
\end{tabular}
\end{table}


\begin{figure}[t]
  \centering
  \includegraphics[width=0.90\linewidth]{tests_landau_hermite/latest/runtime_rhs_scaling.png}
  \caption{Per-RHS-call runtime scaling across $n_{\max}$ (from the latest test run).}
  \label{fig:runtime_rhs_scaling}
\end{figure}

\begin{figure}[t]
  \centering
  \includegraphics[width=0.90\linewidth]{tests_landau_hermite/latest/maxwellian_residual.png}
  \caption{Maxwellian fixed-point residual $\|RHS(M)\|/\|M\|$ across $n_{\max}$.}
  \label{fig:maxw_res}
\end{figure}

\begin{figure}[t]
  \centering
  \includegraphics[width=0.90\linewidth]{tests_landau_hermite/latest/conservation_rates.png}
  \caption{Cross-collision conservation-rate check: total invariant rates computed from the RHS on a random near-Maxwellian state.}
  \label{fig:cons_rates}
\end{figure}

\begin{figure}[t]
  \centering
  \includegraphics[width=0.90\linewidth]{tests_landau_hermite/latest/physics_entropy_free_energy.png}
  \caption{Physics-style monotonicity check: nonlinear KL entropy to instantaneous local Maxwellians (solid) and linearized quadratic free energy about fixed Maxwellians (dashed).}
  \label{fig:entropy_free_energy}
\end{figure}

\begin{figure}[t]
  \centering
  \includegraphics[width=0.90\linewidth]{tests_landau_hermite/latest/rhs_backend_consistency.png}
  \caption{Backend consistency between NumPy and JAX for small $n_{\max}$ where NumPy comparisons are enabled.}
  \label{fig:backend_consistency}
\end{figure}

\begin{figure}[t]
  \centering
  \includegraphics[width=0.90\linewidth]{tests_landau_hermite/latest/linearization_fd_best.png}
  \caption{Finite-difference self-consistency check of the linearization (NumPy, where enabled).}
  \label{fig:lin_fd}
\end{figure}

\begin{figure}[t]
  \centering
  \includegraphics[width=0.90\linewidth]{tests_landau_hermite/latest/linearization_small_perturb_1sp.png}
  \caption{Small-perturbation check: nonlinear vs tangent-linear for the 1-species problem.}
  \label{fig:lin_small_1sp}
\end{figure}

\begin{figure}[t]
  \centering
  \includegraphics[width=0.90\linewidth]{tests_landau_hermite/latest/linearization_small_perturb_2sp.png}
  \caption{Small-perturbation check: nonlinear vs tangent-linear for the 2-species temperature-exchange metric.}
  \label{fig:lin_small_2sp}
\end{figure}

\begin{figure}[t]
  \centering
  \includegraphics[width=0.90\linewidth]{tests_landau_hermite/latest/jax_compile_scaling.png}
  \caption{JAX compile-time scaling across $n_{\max}$ (first call).}
  \label{fig:jax_compile}
\end{figure}

\begin{figure}[t]
  \centering
  \includegraphics[width=0.90\linewidth]{tests_landau_hermite/latest/runtime_integrate_short.png}
  \caption{Short integration benchmark (NumPy vs JAX where enabled).}
  \label{fig:integrate_short}
\end{figure}

\clearpage
\section{Interpreting results and limitations}

\subsection{Positivity}
Hermite spectral truncations do \emph{not} guarantee pointwise positivity of the reconstructed distribution
function, even if the underlying physical solution is nonnegative. This is a well-known spectral/Gibbs
phenomenon. To keep the plotted slices physically meaningful, the default Fig.~1 one-species IC is built
from a manifestly nonnegative two-stream mixture and is optionally ``damped'' in high modes to remain
nonnegative on diagnostic grids.

\subsection{Monotonicity}
For the nonlinear operator, $\mathcal{D}(t)$ (KL divergence to the instantaneous local Maxwellian) is the
referee-proof diagnostic. For the linearized evolution, the correct monotone quantity is the quadratic
free energy $\mathcal{F}(t;M)$ about the fixed Maxwellian background, not KL to a time-dependent Maxwellian.

\subsection{Truncation and parameter choices}
The separation accuracy is controlled by:
\begin{itemize}[nosep]
  \item $n_{\max}$: Hermite resolution (spectral truncation),
  \item $Q$: quadrature nodes in the SOE separation,
  \item \code{maxK}: maximum Hankel index in the Coulomb moment tables.
\end{itemize}
The test suite includes Q/\code{maxK} sweeps and Maxwellian fixed-point residuals to help select safe
defaults.

\appendix

\section{Full CLI help}\label{app:cli}
For completeness, this is the exact output of \code{python landau\_hermite\_jax.py -h} captured at the
time this companion was generated:
\VerbatimInput[fontsize=\small]{landau_hermite_jax_help.txt}

\section{References (selected)}
\begin{thebibliography}{99}

\bibitem{Landau1936}
L.~D.~Landau, ``Kinetic equation for the Coulomb effect,'' \emph{Physikalische Zeitschrift der Sowjetunion} (1936).

\bibitem{Rosenbluth1957}
M.~N.~Rosenbluth, W.~M.~MacDonald, and D.~L.~Judd, ``Fokker--Planck equation for an inverse-square force,''
\emph{Physical Review} \textbf{107} (1957).

\bibitem{HelanderSigmar}
P.~Helander and D.~J.~Sigmar, \emph{Collisional Transport in Magnetized Plasmas}, Cambridge Univ. Press (2002).

\bibitem{Villani2002}
C.~Villani, ``A review of mathematical topics in collisional kinetic theory,'' in \emph{Handbook of Mathematical Fluid Dynamics}, Vol.~I (2002).

\bibitem{ShuOsher1988}
C.-W.~Shu and S.~Osher, ``Efficient implementation of essentially non-oscillatory shock-capturing schemes,''
\emph{Journal of Computational Physics} \textbf{77} (1988).

\bibitem{GottliebShuTadmor2001}
S.~Gottlieb, C.-W.~Shu, and E.~Tadmor, ``Strong stability-preserving high-order time discretization methods,''
\emph{SIAM Review} \textbf{43} (2001).

\bibitem{Oseledets2011}
I.~V.~Oseledets, ``Tensor-train decomposition,'' \emph{SIAM Journal on Scientific Computing} \textbf{33} (2011).

\bibitem{Hackbusch2012}
W.~Hackbusch, \emph{Tensor Spaces and Numerical Tensor Calculus}, Springer (2012).

\bibitem{Schollwoeck2011}
U.~Schollw\"ock, ``The density-matrix renormalization group in the age of matrix product states,''
\emph{Annals of Physics} \textbf{326} (2011).

\bibitem{Orus2014}
R.~Or\'us, ``A practical introduction to tensor networks: Matrix product states and projected entangled pair states,''
\emph{Annals of Physics} \textbf{349} (2014).

\bibitem{GolubWelsch}
G.~H.~Golub and J.~H.~Welsch, ``Calculation of Gauss quadrature rules,'' \emph{Mathematics of Computation} \textbf{23} (1969).

\bibitem{Boyd}
J.~P.~Boyd, \emph{Chebyshev and Fourier Spectral Methods}, 2nd ed., Dover (2001). (Background on spectral truncation/oscillations.)

\bibitem{Grad}
H.~Grad, ``On the kinetic theory of rarefied gases,'' \emph{Communications on Pure and Applied Mathematics} \textbf{2} (1949).

\bibitem{BeylkinMonzon}
G.~Beylkin and L.~Monz\'on, ``Approximation by exponential sums revisited,'' \emph{Applied and Computational Harmonic Analysis} (2005).

\bibitem{JAX2018}
J.~Bradbury et al., ``JAX: composable transformations of Python+NumPy programs,'' (2018). \url{https://github.com/jax-ml/jax}

\end{thebibliography}

\end{document}
